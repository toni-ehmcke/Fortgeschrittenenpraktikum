\section{Durchführung}

\subsection{Strahlenschutzkontrollmessungen}
Im Ersten Arbeitsschritt haben wir die Dosen an verschiedenen Stellen im Versuchsraum vermessen. Dabei ist zu beachten, dass das verwendete Dosimeter Messwerte über einen bestimmten Zeitraum (ca. 30 Sekunden) mittelt. Außerdem handelt es sich um die Dosen, welche am morgen zum Arbeitsbeginn (Tabelle \ref{dft:Arbeitsplatz}) gemessen wurden. 
Im Zweiten Schritt haben wir mittels vier der BeO-Sonden, die von der Betreuerin am 21.10.2015 um 16:30 Uhr im Raum verteilt deponiert wurden, am Nachmittag des Versuchstages (23.10.15) um 15 Uhr ausgelesen. (Tabelle \ref{dft:Raum}) \\
	
\minipanf
	\begin{center}
					\begin{tabular}{l|c|c|c}
								\textbf{Ort} & \textbf{\.D} [$\mu$Sv/h] & \textbf{\.D} [mSv/a]  & \textbf{t$_{max}$} [h]\\ 
						\hline  Arbeitsplatz (vor PC) & 0,15  &   1,31 & 109,59\\ 
								50cm hinter Quelle    & 0,77  &   6,75 &  21,35\\ 
								50cm rechts der Quelle& 2,28  &  19,97 &   7,21\\ 
								50cm links der Quelle & 2,41  &  21,11 &   6,82\\ 
								50cm vor der Quelle   & 6,34  &  55,53 &   2,59\\ 
								8cm hinter der Quelle & 5,20  &  45,55 &   3,16\\ 
								8cm vor der Quelle    & 61,20 & 536,11 &   0,27\\
						\hline  50cm hinter Quelle    & 1,06  &   9,29 &  15,51\\
								50cm rechts der Quelle& 2,68  &  23,48 &   6,13\\ 
								50cm links der Quelle & 3,22  &  28,71 &   5,11\\
					\end{tabular}	
					\captionof{table}{Messtabelle für Dosisleistung an verschiedenen Orten und der daraus resultierenden maximalen Verweildauer pro Tag}
					\label{dft:Arbeitsplatz}
					\vspace{8mm}
					\begin{tabular}{l|c|c|c}
								\textbf{Ort} & \textbf{BeO-Nr.} & \textbf{LS} [mVs] & \textbf{D} [mGy] \\
						\hline  \"{}Mitfahrer vor Quelle\" & 003337 & 14,19 & 0,5028 \\
								Auf Steckerleiste:                &        &       &        \\
								 \hspace{3mm} 1,8m vor Quelle     & 003353 & 1,73  & 0,0587 \\
								 \hspace{3mm} neben Quelle        & 003350 & 1,14  & 0,0437 \\
								 \hspace{3mm} hinter Quelle       & 003376 & 0,19  & 0,0066 \\
					\end{tabular}
					\captionof{table}{BeO-Sensordaten, die über zwei Tage im Versuchsraum gesammelt wurden}
					\label{dft:Raum}
                    \vspace{8mm}
	\end{center}
\minipend
	
Die BeO-Nummer ist die Identifikationsnummer des Sensors. Dadurch lassen sich alle Daten zurückverfolgen, da sich die Auswertesoftware die Nummern und die Verwendung der Sensoren merkt. Durch die IDs können dank einer intelligenten Software auch Fehler vermieden werden (eventuell verfrühte Ausleuchtung etc.) LS bedeutet hier das gemessene Lichtsignal.
		
\subsection{Winkelabhängigkeit einer Ionisationskammer} \label{sec:Ionkammer}

Nun wenden wir uns der Messungen bezüglich der Winkelabhängigkeit einer zylindrischen Ionenstrahlkammer zu. Die Messung startet bei 180\textdegree und dauert pro Winkel jeweils 60 Sekunden.
Es handelt sich hier um eine aktive Sonde, die über ein Leichtleiterkabel an ein Anzeigegerät, welches alle gemessenen Werte innerhalb des Messzeitintervalls einer Messung automatisch aufsummiert, angeschlossen wurde.

\vspace{5mm}
	\begin{center}
		\begin{tabular}{c|c|c|c|c|c}
				\textbf{$\alpha$} [\textdegree] & \textbf{d} [m] & \textbf{D$_0$} [$\mu$Gy] & \textbf{\.D$_0$} [$\mu$Gy/s] & \textbf{\.D} [$\mu$ Gy/s] & $\Delta$ \textbf{\.D} [$\mu$Gy/s]\\ 
		\hline	180 & 0,50 & 18,29 & 0,3048 & 0,3089 & 0,00073 \\ 
				150 & 0,50 & 18,26 & 0,3043 & 0,3084 & 0,00073 \\ 
				120 & 0,50 & 17,94 & 0,2990 & 0,3030 & 0,00072 \\ 
				90  & 0,50 & 17,66 & 0,2943 & 0,2982 & 0,00072 \\ 
				60  & 0,50 & 17,49 & 0,2915 & 0,2954 & 0,00071 \\ 
				30  & 0,50 & 17,19 & 0,2865 & 0,2903 & 0,00071 \\ 
				0   & 0,50 & 17,24 & 0,2873 & 0,2911 & 0,00071 \\ 
				330 & 0,50 & 17,38 & 0,2897 & 0,2935 & 0,00071 \\ 
				300 & 0,50 & 17,52 & 0,2920 & 0,2959 & 0,00071 \\ 
				270 & 0,50 & 17,81 & 0,2968 & 0,3008 & 0,00072 \\ 
				240 & 0,50 & 18,13 & 0,3022 & 0,3062 & 0,00073 \\ 
			    210 & 0,50 & 18,26 & 0,3043 & 0,3084 & 0,00073 \\ 
		\hline	180 & 0,50 & 18,39 & 0,3065 & 0,3106 & 0,00073 \\ 
		\end{tabular} 
		\captionof{table}{Messwerte aus der Winkelabhängigkeit der Zylinderkammer}
		\label{dft:Winkel}
	\end{center}
\vspace{5mm}				

Die Drehung der zylindrischen Ionenstrahlkammer wurde um die z-Achse vorgenommen. Dabei sind 180\textdegree gerade die vorgegebene Vorzugsrichtung, in der die Kammer Richtung Quelle zeigt. Dies wurde am Gerät durch eine eine grüne Strichmarkierung angezeigt. Während der Messungen betrugen die Temperatur im Raum $T = (294,75\pm 0,3)K$, der Druck $p = (100,52 \pm 0,01)kPa$ und die Luftfeuchtigkeit 54\%. Dabei sind die Fehler für Temperatur und Druck abgeschätzt. Während die Temperatur im Raum Schwankungen unterworfen war, konnte der Luftdruck nahezu erhalten bleiben. Bezüglich D$_0$ schätzen wir aufgrund des statistischen Charakters während der Messungen inklusive des Digits einen Fehler von 0,03$\mu$Gy ab\\
\.D bezeichnet die Luftdruck-Temperatur-korrigierte Dosisleistung. Sie ergibt sich folgendermaßen aus Formel (\ref{formel:kappa}):

\begin{equation*}
        \dot D = \kappa D_0 \ \text{ mit } \ \kappa = \frac{\rho}{\rho_0} = \frac{T}{T_0} \frac{p_0}{p}
\end{equation*}

Dies führt uns auf folgenden Fehler $\Delta$\.D:

\begin{equation*}
    \Delta \dot D = \dot D \sqrt{\left(\frac{\Delta \kappa}{\kappa}\right)^2 + \left( \frac{\Delta \dot D_0}{\dot D_0}\right)^2} \text{ wobei } \Delta \kappa = \kappa \sqrt{\left(\frac{\Delta T}{T}\right)^2 + \left(\frac{\Delta p}{p}\right)^2}
\end{equation*}

Da wir die Temperatur als nahezu konstant annahmen ist $\frac{\Delta \kappa}{\kappa}$ eine Konstante. $\frac{\Delta \dot D}{\dot D}$ scheint zunächst unbekannt. Es ergibt sich jedoch aus einer nahezu fehlerfreien Zeitmessung und $\dot D = D/(\Delta t)$:

\begin{equation*}
    \frac{\Delta \dot D}{\dot D} =  \frac{\Delta D}{D}
\end{equation*}

Da sich das Zeitintervall $\Delta t$ herauskürzt.

\subsection{Abschätzung der Dosisleistung für $\gamma$-Strahler}

Als nächstes widmeten wir uns der Messung der Dosis in verschiedenen Abständen. Dieses Mal verwendeten wir das BeO$max$-System. Wir verwenden immer vier der Beo-Chips in einer nahezu quadratischen Anordnung zwischen zwei verschieden dicken Plexiglas-Platten. Die dickere Platte zeigt stets von der Quelle weg.
Die BeO's müssen immer mit etwa $\Delta D = 120\ \mu$Gy bestrahlt werden, was bei verschiedenen Abständen $d$ zu verschiedenen Messzeitintervallen führt. Unter Verwendung von (\ref{eq:dosisleistung}) und (\ref{eq:aktivitaet}) erhalten wir als Abschätzung für die Bestrahlzeit:
\begin{equation}
		\Delta t 	= \frac{d^2 \cdot \Delta D}{A(t) \cdot \Gamma} 
					= 2^{\frac{t}{T_{1/2}}} \cdot \frac{d^2 \cdot \Delta D}{A_0 \cdot \Gamma} 
\end{equation}

\vspace{5mm}
\minipanf
	\begin{center}
		\begin{tabular}{c|c}
				   \textbf{d} [m] & \textbf{$\Delta$t} [min] \\ 
		\hline     0,2 &  0,78 \\ 
				   0,3 &  1,76 \\ 
				   0,4 &  3,12 \\ 
				   0,5 &  4,88 \\ 
				   0,6 &  7,03 \\ 
				   0,7 &  9,56 \\ 
				   0,8 & 12,49 \\ 
				   0,9 & 15,81 \\ 
				   1,0 & 19.51 \\  
		\end{tabular}
		\captionof{table}{ungefähre Bestrahlungsdauer in Abhängigkeit vom Abstand}
		\label{dft:Zeiten} 
	\end{center}
\minipend
\vspace{5mm}

Diese Zeiten wurden nur als grobe Orientierungspunkte verwendet und sind aufgrund verschiedener Faktoren nicht genau eingehalten worden.

\subsection{Dosismessung mit BeO$max$ und Abstandsquadratsgesetz}
Nun kommen wir zu den Messergebnissen für verschiedene Abstände. Verwendet wurden vier in der Mitte einer Plexisglas-Platte aufgebrachte BeO-Sensoren. Diese wurden für die in Tabelle \ref{dft:Zeiten} genannten Zeiten bestrahlt.\\ 
Zunächst führen wir für 4 BeOs eine Kalibrationsmessung durch, indem wir das Ansprechvermögen $\epsilon = (LS - LS_0)/D_{ref}$ bestimmen.
In Abschnitt \ref{sec:Ionkammer} haben wir dafür die Dosisleistung bei $d=0,5\ m$ und $\alpha =$ 180\textdegree \ mit $\dot{D}_{ref} = (0,309 \pm 0.001)\mu Gy /s$ bestimmt. Es ergibt sich durch die längere Messzeit von $\Delta t = 292,8\ s$ eine Dosis von $D_{ref} = \dot{D}_{ref} \cdot \Delta t = (90,5 \pm 0,4)\mu Gy$. Wobei ein systematischer Fehler bei der Zeitmessung von $\Delta(\Delta t) = 0,5s$ als Reaktionszeit angenommen wurde. Wir wiederholen nun die Messung mit dem Beo\textit{max}-System und bestimmen das Ansprechvermögen:\\

\minipanf
	\begin{center}	
		\begin{tabular}{c|c|c|c|c}
					\textbf{d} [m] & BeO-Nr. & \textbf{LS$_0$} [mVs] & \textbf{LS} [mVs] & \textbf{$\epsilon$} [$\frac{mVs}{\mu Gy}$] \\
			\hline  
								& 003305 & 0,66 & 2,84 & 0.024\\
						0,5		& 003398 & 0,85 & 3,68 & 0.031\\
								& 003372 & 0,77 & 3,34 & 0,028\\
								& 003312 & 0,81 & 3,18 & 0,026\\								
		\end{tabular}
		\captionof{table}{Bestimmung des Ansprechvermögens}
	\end{center}
\minipend
\vspace{3mm}

\minipanf
	\begin{center}	
		\begin{tabular}{c|c|c|c|c}
				\textbf{d} [m] & BeO-Nr. & \textbf{LS$_0$} [mVs] & \textbf{LS} [mVs] & \textbf{D} [mGy] \\
			\hline  
								& 003305 & 0,65 & 1,99 & 0,0853\\
						0,3		& 003398 & 0,86 & 2,52 & 0,0863\\
								& 003372 & 0,79 & 2,28 & 0,0850\\
								& 003312 & 0,84 & 2,09 & 0,0846\\
			\hline  
								& 003302 & 0,93 & 2,09 & 0,0883\\
						0,4		& 003332 & 0,70 & 2,59 & 0,0899\\
								& 003395 & 1,04 & 2,09 & 0,0870\\
								& 003325 & 2,16 & 2,99 & 0,1415\\
			\hline  
                                & 003305 & 0,66 & 2,84 & 0,0955\\
                        0,5     & 003398 & 0,85 & 3,86 & 0,0933\\
                                & 003312 & 0,82 & 3,18 & 0,1029\\
                                & 003372 & 0,77 & 3,34 & 0,0956\\
            \hline
                                & 003397 & 0,68 & 2,01 & 0,0954\\
                        0,6     & 003364 & 0,80 & 2,92 & 0,0957\\
                                & 003388 & 0,67 & 3,32 & 0,1134\\
                                & 003338 & 0,62 & 3,13 & 0,0932\\
            \hline
								& 003312 & 0,82 & 2,48 & 0,1005\\
						0,7		& 003305 & 0,69 & 2,36 & 0,1008\\
								& 003398 & 0,88 & 3,00 & 0,1025\\
								& 003372 & 0,78 & 2,69 & 0,1002\\
			\hline  
								& 003397 & 0,66 & 2,36 & 0,1011\\
						0,8		& 003388 & 0,72 & 2,73 & 0,1025\\
								& 003364 & 0,74 & 2,39 & 0,1078\\
								& 003338 & 0,61 & 2,70 & 0,1004\\
			\hline  
								& 003325 & 2,18 & 2,15 & 0,1019\\
						0,905   & 003395 & 1,05 & 2,38 & 0,0988\\
								& 003332 & 0,69 & 3,05 & 0,1056\\
								& 003302 & 0,91 & 2,50 & 0,1056\\
			\hline  
								& 003372 & 0,80 & 2,71 & 0,1010\\
						1,0		& 003312 & 0,82 & 2,57 & 0,1038\\
								& 003398 & 0,86 & 3,08 & 0,1051\\
								& 003305 & 0,67 & 1,83 & 0,0785\\
								
		\end{tabular}
		\captionof{table}{Sensor-ID, Lichtsignal der Nullmessung, Lichtsignal nach Bestrahlung und Dosis}
        \label{dft:osl}
	\end{center}
\minipend
\vspace{5mm}

Die Dosiswerte der beiden Abstände $d = 0,5m$ und $d = 0,6m$ wurden leider nicht vermessen. Wir haben diese allerdings von Hand über die vom PC bekannten Ansprechnvermögen $\epsilon$ nachträglich ausgerechnet. \\
Damit haben wir nun die Dosisleistung für die einzelnen Abstände bestimmt, um das Abstandsquadratsgesetz zu überprüfen:

\minipanf
	\begin{center}	
		\begin{tabular}{c|c|c|c|c|c|c}
					d [m] & $\overline{D}$ [$\mu Gy$] & $\Delta D_{sys}$ [$\mu Gy$] & $\sigma(\overline{D})$ [$\mu Gy$] & $\Delta t$ [s] & $\dot{D}$ [$\mu Gy / s$] & $\Delta\dot{D}$ [$\mu Gy / s$] \\
		\hline	0.3 & 85 & 2.6 & 0.36 & 112 & 0.437 & 0.016 \\
				0.4	& 88 & 2.6 & 0.84 & 195 & 0.453 & 0.018 \\
				0.5 & 97 & 2.8 & 2.08 & 293 & 0.331 & 0.017 \\
			    0.6	& 99 & 2.8 & 4.69 & 422 & 0.236 & 0.018 \\
			    0.7 & 101 & 2.8 & 0.51 & 576 & 0.175 & 0.006 \\
			    0.8	& 103 & 2.8 & 1.67 & 750 & 0.137 & 0.006 \\
				0.9 & 103 & 2.8 & 1.64 & 975 & 0.106 & 0.005 \\
				1.0 & 97 & 2.8 & 6.26 & 1170 & 0.083 & 0.008				
		\end{tabular}
		\captionof{table}{Dosisleistung in Abhängigkeit vom Abstand}
		\label{dft:Abstandsquadrat}
	\end{center}
\minipend
\vspace{3mm}

Dabei wurden folgende Relationen verwendet:
\begin{equation*}
	\Delta D_{sys} = 1,5 \mu Gy + 0,013 \cdot D \\
\end{equation*}
\begin{equation*}
	\Delta \dot{D} = \dot{D} \cdot \sqrt{\left(\frac{\Delta (\Delta t)}			{\Delta t}\right)^2 + \left(\frac{\Delta D}{D} \right)^2} + \dot{D} 			\cdot \frac{\sigma(\overline{D})}{D}
\end{equation*}
Wobei $\Delta t = 1\ s$ als Reaktionszeit, um den Stopfen auf die Quelle zu packen, großzügig abgeschätzt wurde. Es wurden die Stoppuhrzeiten notiert - die vorher theoretisch berechneten Zeiten dienten nur als grobe Orientierung.

\subsection{Winkelabhängigkeit der kollimierten Quelle}

Zuletzt interessiert uns die Strahlaufweitung unserer Quelle. Dafür nahmen acht BeO-Sonden, die wir nebeneinander in einer Reihe auf die 16cm breite Plexiglas-Halterung legten. Die Sonden diesmal nicht zentriert, sondern von Zentrum rechtsseitig versetzt worden. Der Abstand zum Sensor um Zentrum des Strahls beträgt $d = 0,3$m. Daraus ergibt sich eine ungefähre Bestrahldauer von $\Delta t = 5$min. Die Zeit ist größer als bei den Abstandsmessungen zuvor, da sich die Abstände mit wachsendem Winkel verändert.

\vspace{5mm}
\minipanf
	\begin{center}
		\begin{tabular}{c|c|c|c|c|c}
				\textbf{BeO-Nr.}& \textbf{s} [m] & \textbf{LS$_0$} [mVs] & \textbf{LS} [mVs] & \textbf{D} [$\mu Gy$] & $\Delta$ \textbf{D} [$\mu Gy$] \\
		 \hline 003338 &  0.01 & 0,62 & 6,17 & 229 & 4,5 \\
				003302 &  0.03 & 0,92 & 5,43 & 229 & 4,5 \\
				003337 &  0.05 & 0,48 & 6,36 & 225 & 4,4 \\
				003364 &  0.07 & 0,74 & 4,98 & 225 & 4,4 \\
				003353 &  0.09 & 0,44 & 5,10 & 173 & 3,8 \\
				003388 & 0.11	& 0,68 & 2,15 & 81 & 2,6 \\
				003332 & 0.13 & 0,73 & 1,37 & 48 & 2,1 \\
				003376 & 0.15 & 0,59 & 0,74 & 26 & 1,8 \\
		\end{tabular}
		\captionof{table}{Messung der Strahlaufweitung der Quelle durch den Kollimator}
        \label{dft:Aufweitung}
	\end{center} 
\minipend
\vspace{5mm}

Die Fit-Parameter haben wir mit 


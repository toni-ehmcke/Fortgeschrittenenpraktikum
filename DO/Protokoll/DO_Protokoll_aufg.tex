\section{Aufgabenstellung}
Im Experiment werden wir die folgenden Aufgabenstellungen bearbeiten:
\subsection{Strahlenschutzkontrollmessungen}
Zunächst bestimmen wir die Dosisleistung in den Bestrahlungsräumen bei geschlossener und geöffneter Quelle an Positionen, an denen später gemessen wird und an denen wir uns häufig aufhalten werden. Dies hat den Zweck der Überwachung der durch die Experimentatoren aufgenommenen Dosis und des Strahlungshintergrundes, welcher die spätere Dosismessung beeinflussen kann. Es soll eine Skizze des Messraumes mit den entsprechenden Dosiswerten entstehen.

\subsection{Ionisationskammer}
Mit Hilfe einer Ionisationskammer sollen die absoluten Dosiswerte an Referenzpunkten gemessen werden, an denen wir später die OSL-Dosimeter kalibrieren werden. Weiterhin untersuchen wir die Richtungsabhängigkeit der Ionisationskammer. Dafür nutzen wir eine Kugelkammer ($V_K = 27,9\ cm^3$) und eine Stielkammer ($V_S = 30,0\ cm^3$).

\subsection{Abschätzung der Dosisleistung für $\gamma$-Strahler}
Für vorgegebene Nuklide ($^{241}Am,\ ^{60}Co,\ ^{137}Cs$) soll die Dosisleistung
\begin{equation}
	\dot{D}=\frac{A \cdot \Gamma}{d^2}
\end{equation}
abgeschätzt werden. Dabei sind:
\begin{center}
\begin{minipage}{.9\textwidth}
d: Abstand zur Quelle\\
	A: Aktivität der Quelle am Versuchstag(22.10.2015), \\
	Referenzwert $A(27.02.2008) = 5,0\ GBq$\\
	$\Gamma$: Dosisleistungskonstante
	
	%----------------------------------------Tabelle einfügen---------------------------------
\end{minipage}
\end{center}

\subsection{OSL-Dosimetrie}
Wir werden anschließend mit Hilfe des BeO\textit{max}-OSL-Dosimetriesystems eine Dosismessung vornehmen. Da dieses Messsystem ein relatives Vorgehen verlangt, wird die Messung in zwei Schritten vorgenommen: 1) der Kalibrierung mit einer bekannten Dosis (siehe 2.2) und 2) der Dosisbestimmung einer unbekannten Dosis.
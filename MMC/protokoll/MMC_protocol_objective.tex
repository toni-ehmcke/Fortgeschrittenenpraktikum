\section{Aufgabenstellung}
Im Versuch wurden folgende Aufgabenstellungen bearbeitet:
\subsection{Radon-Transformation als Rekonstruktionsverfahren: Ein Rechenbeispiel}
Es soll anhand einer willkürlich vorgegebenen theoretischen Quellverteilung in Matrixform die Radon-Transformation nachvollzogen und das Ergebnis in einer Graustufendarstellung visualisiert werden.

\subsection{Kalibriermessungen}
	\subsubsection{Messung einer Quelle bekannter Aktivität bei mittiger Quellposition}
	Es soll das Energie- und Koinzidenzzeitfenster für die spätere Messung durch die Verwendung einer Referenzquelle in mittiger Position bestimmt werden. Weiterhin wird die Koinzidenzauflösungszeit, die Koinzidenznachweiseffektivität und der Anteil zufälliger Koinzidenzen ermittelt. Weiterhin sollen Schwerpunktdiagramme entstehen, welche verdeutlichen, auf welche Bereiche der Detektormatrix die ionisierende Strahlung fällt.
	\subsubsection{Messung bei Positionen direkt an den Detektoren}
	Die obige Messung wird noch einmal für die Position der Quelle direkt vor den Detektoren wiederholt, um einen Vergleich des Schwerpunktdiagramms zu erhalten und um den Detektorabstand zu bestimmen
	
\subsection{Tomografische Messungen}
	\subsubsection{Messung einer Quellkonfiguration, Phantom isotroper Dichteverteilung}
	Anhand einer unbekannten  Quellverteilung mit gleichmäßiger Dichteverteilung soll die Bildentstehung nachvollzogen und der Einfluss verschiedener Filter untersucht werden. Anschließend soll ein geeigneter Filter gewählt werden, mit dessen Hilfe die Quellpositionen, die Quellaktivitäten und die Ortsauflösung des PET-Scans ermittelt werden.
	\subsubsection{Messung mit einer Punktquelle, Phantom an-/isotroper Dichteverteilung}
	Eine Punktquelle wird in der Mitte des Probenhalters platziert, um den Einfluss der Dichteverteilung des umgebenden Materials zu untersuchen, um ein Modell für den menschlichen Körper zu erhalten, welcher ebenfalls eine anisotrope Dichteverteilung aufweist. Auf Grundlage dieser Messung soll eine Korrekturfunktion ermittelt werden, die diese Dichteschwankungen ausgleicht.
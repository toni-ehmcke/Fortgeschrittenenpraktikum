\section{Durchführung}
\subsection{Theoretischer Teil}

\subsection{Kalibriermessungen}
    \subsubsection{Messung einer Quelle bekannter Aktivität bei mittiger Quellposition}
       Zunächst haben wir eine Quelle in mittigem Abstand zu den beiden Detektoren vermessen. Die Quelle hatte am 29.10.2015 eine Aktivitiät $A = \unit[1,02]{MBq}$.\\
       Daraus sollten die Energiefenster der beiden Detektoren sowie das Koinzidenzzeitfenster bestimmt werden. Die Energiepeaks werden um $\unit[511]{keV}$ erwartet, was einem
       Kanal von 2402 für Detektor A und 1868 für Detektor B entspricht. Das Zeitfenster erwarten wir aufgrund technischer Effekte (bspw. Totzeit) und einer gewollten
       erhöhten Verzögerung im Bereich einiger zehn Nanosekunden.  
       \vspace{2mm}
        
       \begin{longtable}{p{6cm}p{6cm}l}
               \minipanf 
                   \includegraphics[width=1.2\textwidth, height=0.225\textheight]{pic/Efenster_DetA_M.png}
                   \label{dfd:EdetA_M}
               \minipend
               &
               \hspace{9mm}
               \minipanf 
                   \includegraphics[width=1.2\textwidth, height=0.225\textheight]{pic/Efenster_DetB_M.png}
                   \label{dfd:EdetB_M}
               \minipend\\
               \multicolumn{2}{c}{\hspace{1.5cm}
                   \includegraphics[width=0.7\textwidth, height=0.3\textheight]{pic/T_M_dia.png}
                   \label{dfd:T_M}}\\                        
       \end{longtable}
       \captionof{table}{Kalibrationsmessung bei Quelle mittig zwischen den Detektoren A und B}
       \label{dft:kalim}
       \ \\
       Wie man an den oberen beiden Diagrammen von Tabelle \ref{dft:kalim} sieht, stimmt diese Erwartung mit einer leichten Verschiebung nach rechts gut überein. Dies kann
       dem statistischen Charakter von Kernzerfällen und den Wegen (verändert ua. durch Compton-Streuung für Photonen, freiwerdende Bremsstrahlung für Positronen), die die 
       Positronen und letztlich die Photonen zurücklegen zugeordnet werden. Andererseits unterliegen die Detektormessungen und die Umrechnung der Kanäle in Energien einem 
       Fehler, da die Kanäle diskret sind und nur bestimmte Energien detektieren können, was dazu führt, dass Energien zwischen zwei Kanälen verloren gehen.
       Die Umrechnung der Kanäle in Energien und Zeit erfolgt mittels folgender, uns gegebener Formeln des Musters $G = a + b \cdot K$, wobei K die Kanalnummer meint und a
       bzw. b Parameter sind:
       \begin{gather*}
           E_A = (81\pm 31) + (0,179\pm 0,012)K_A\\
           E_B = (100\pm 40) + (0,220\pm 0,021)K_B\\
           \Delta t = (-0,014\pm 0,0192) + (0,0483\pm 0,00002)K_t
       \end{gather*}
       
       Die Fehler für die Umrechnung von Kanälen in physikalische Einheiten lassen sich leicht über die gauß'sche Fehlerfortpflanzung bestimmten und belaufen sich betreffend 
       der Diagramme in \ref{dft:kalim} auf:\\
       $$ \Delta(\Delta t) = \unit[0,03]{ns} ; \Delta E_A = \unit[44,84]{keV}; \Delta E_A = \unit[44,00]{keV} $$
       Um Messungen vorzunehmen, müssen jetzt die Fenster für Energie und Koninzidenzzeit bestimmt werden. Dies geschieht über Gauß-Fits in den Diagrammen von Tabelle \ref{dft:kalim}.
       Dann bestimmt man das FWHM und zieht es von den Maxmumpositionen ab. Damit ergeben sich die Fenster zu:
       $$ E_A \in \left[ 2217, 3101\right], E_B \in \left[ 1629, 2264 \right], \Delta t \in \left[x\right] $$ 
       Daraus ergibt sich eine Koinzidenzauflösungszeit von $\tau = \unit[4,77]{ns}$, welche auf eine Abschätzung der zufälligen Koinzidenzen führt. Diese können mit verschiedenen
       Gedankengängen abgeschätzt werden. Wir haben die Wahrscheinlichkeit, dass beim Zerfall der $^{22}\unit{Na}$ Positronen emittiert werden gegeben. 


    \subsubsection{Messung bei Positionen direkt an den Detektoren}
        \begin{longtable}{p{6cm}p{6cm}l}
            \minipanf 
                \includegraphics[width=1.2\textwidth, height=0.225\textheight]{pic/T_A_dia.png}
                \label{dfd:T_A}
            \minipend
            &
            \hspace{9mm} 
            \minipanf
                \includegraphics[width=1.2\textwidth, height=0.225\textheight]{pic/T_B_dia.png}
                \label{dfd:T_B}
            \minipend \\
            \minipanf
                \includegraphics[width=1.2\textwidth, height=0.225\textheight]{pic/Efenster_DetA_A.png}
                \label{dfd:EdetAA}
            \minipend
            &
            \hspace{9mm} 
            \minipanf 
                \includegraphics[width=1.2\textwidth, height=0.225\textheight]{pic/Efenster_DetA_B.png}
                \label{dfd:EdetBA}
            \minipend \\
            \minipanf 
                \includegraphics[width=1.2\textwidth, height=0.225\textheight]{pic/Efenster_DetB_A.png}
                \label{dfd:EdetAB}
            \minipend
            &
            \minipanf 
                \hspace{9mm}
                \includegraphics[width=1.2\textwidth, height=0.225\textheight]{pic/Efenster_DetB_B.png}
                \label{dfd:EdetBB}
            \minipend \\            
        \end{longtable}
        \captionof{figure}{Gegenüberstellung der Messungen mit der Quelle an Det. A (links) und Det. B (rechts)}
        
    \subsubsection{Schwerpunktsdiagramme}
    Als nächstes sind die Schwerpunktsdiagramme zu betrachen. Es wurden drei von Detektor A erstellt. Einmal als die Quelle an Detektor B positioniert wurde, danach an Detektor A und einmal mittig zwischen beiden.\\
    In Abbildung \ref{dfp:SPDdetB} kann gegenüber \ref{dfp:SPDMitte} ein schwächeres Muster. Gegenüber Abbildung \ref{dfp:SPDMitte} ist sogar deutlich, die Kristallstruktur erkennbar.
    
    \begin{tabular}{p{5cm}p{5cm}p{5cm}c}
        \includegraphics[width=0.3\textwidth, height=0.2\textheight]{pic/Einzelfenster_Bilder/fenster_DetAanB.png}
        \captionof{figure}{Messung bei Quelle an Detektor B}
        \label{dfp:SPDdetB}
        &
        \includegraphics[width=0.3\textwidth, height=0.2\textheight]{pic/Einzelfenster_Bilder/fenster_DetAMitte.png}
        \captionof{figure}{Messung bei Quelle in der Mitte}
        \label{dfp:SPDdetA}
        &
        \includegraphics[width=0.3\textwidth, height=0.2\textheight]{pic/Einzelfenster_Bilder/fenster_DetAanA.png}
        \captionof{figure}{Messung bei Quelle an Detektor A}
        \label{dfp:SPDMitte}
    \end{tabular}
        
    \subsection{Tomografische Messungen}
    	\subsubsection{Messung einer Quellkonfiguration, Phantom isotroper Dichteverteilung}
            \textbf{Hauptversuch}\\
            
            \textbf{Untersuchung des Einflusses verschiedener Filter}\\
            
            \begin{tabular}{p{6cm}p{6cm}c}
                            \minipanf 
                                \makebox[\linewidth]{\includegraphics[width=2\textwidth, height=0.225\textheight]{pic/3dPlotTomographRamp.png}}
                                
                            \minipend
                            &
                            \hspace{3mm} 
                            \minipanf
                                \includegraphics[width=2.2\textwidth, height=0.25\textheight]{pic/3dPlotTomographUngef.png}
                            \minipend \\               
             
             \end{tabular} 
            \captionof{figure}{Gefilterte und Ungefilterte Rückprojektion der Aktivitätsverteilung}
            \label{dfd:AKV}
            \ \\
            
            \textbf{Quantitative Auswertung}
            
            \begin{tabular}{p{12cm}	p{0.5\textwidth}}            	
            	\minipanf
            		Zunächst werden die Positionen $(x_i,y_i) (i = 1,2,3)$ der 3 Quellen im verschlossenen Plastikbehältnis bestimmt. Dafür wird die in Abbildung (\ref{dfd:AKV}) visualisierte Rückprojektion $N(x,y)$ verwendet, die durch Auslesen der in \texttt{Matrix\_reco.txt} enthaltenen Messwertmatrix entstanden ist. Der erste Eintrag sei als Koordinatenursprung gewählt. 1 BIN des Rekonstruktionsrasters entspricht $3,375\ \unit{mm}$. Die Positionen der Quellen werden mit den lokalen Maxima $N(x_i,y_i)$ der Aktivitätsverteilung identifiziert.\\
            		Anschließend quantifiziert man die Aktivität jeder einzelnen Quelle, indem man die rückprojizierten Verteilung über einen kleinen Bereich um die Peaks mittelt. Bezeichne diesen Mittelwert mit $\bar{N}(x_i,y_i)$. Im Rahmen dieser Auswertung wurde ein quadratischer Bereich gewählt, in welchem Werte anzutreffen waren, die in der Nähe des FWHM (=Full Width Half Maximum) lagen. Dieses Vorgehen wird durch die nebenstehende Abbildung visualisiert.
            	\minipend            
            	&
            	\begin{minipage}[c]{\textwidth}
                	\includegraphics[width=0.25\textwidth, height=0.20\textheight]{pic/Skizze_Mittelung.png}
                \end{minipage}
            \end{tabular}\\
            
            Mittels einfacher Verhältnisbildung können unter Vorgabe einer Referenzaktivität $A_{ref}$ nun unbekannte Aktivitäten innerhalb der Verteilung berechnet werden. Dabei wurde die stärkste Aktivität mit $A_0 \equiv A(t_0 = \textrm{01.02.2010}) = (363 \pm 11)\ \unit{kBq}$ angegeben. Mit dem Aktivitätsgesetz kann man nun berechnen:
            
            \begin{equation}
            	A_{ref} \equiv A(t =\textrm{29.10.2015}) = A_0 \cdot \left( \frac{1}{2}\right)^{\frac{t-t_0}{T_{1/2}}} = (79 \pm 3)\ \unit{kBq}
            \end{equation}\\
            Wobei die Halbwertszeit $T_{1/2}(^{22}\textrm{Na}) = (2,6027 \pm 0,0010)\ \unit{a}$ verwendet wurde, sowie folgende Fehlerformel:
            \begin{equation}
                \left(\frac{\Delta A_{ref}}{A_{ref}}\right)^2 = \left(\frac{\Delta A_0}{A_0}\right)^2 + \left(\ln(2) \cdot \frac{\Delta T_{1/2}}{T_{1/2}}\right)^2
            \end{equation}\\   
            Bezeichnet man $A_{ref} \propto \bar{N}_{ref} \equiv \bar{N}(x_1,y_1)$ als rückprojizierte Aktivität der Referenzquelle, so erhält man für die unbekannten Aktivitäten $A_i \propto \bar{N}(x_i,y_i)$:
            \begin{equation}
            	A_i = A_{ref} \cdot \frac{\bar{N}(x_i,y_i)}{\bar{N}_{ref}}
            \end{equation}    
            \begin{equation}
                 \left(\frac{\Delta A_i}{A_i}\right)^2 = \left(\frac{\Delta A_{ref}}{A_{ref}}\right)^2 + \left(\frac{\Delta \bar{N}(x_i,y_i)}{\bar{N}(x_i,y_i)}\right)^2 + \left(\frac{\Delta \bar{N}_{ref}}{\bar{N}_{ref}}\right)^2
             \end{equation}\\   
             Hierbei wurden die Fehler der rückprojizierten Aktivitäten als Standardabweichungen des Mittelwertes gesetzt, die sich beim obigen Mittelvorgang ergab: $\Delta \bar{N}(x_i,y_i) = \sigma(\bar{N})$. Die systematischen Fehler des PET-Scanners waren leider nicht bekannt. Zusammenfassend ergeben sich folgende Resultate:  
            
        \subsubsection{Messung einer Quellkonfiguration, Phantom isotroper Dichteverteilung}
          \textbf{Hauptversuch}\\
          Als nächsten wurde eine Messung mit unbekannter Quellverteilung gestartet. Die Energie- und das Zeitfenster entsprechen den oben bestimmten Intervallen.\\ \ \\

            \begin{longtable}{p{7cm}p{7cm}c}
                \textbf{ungefilterter Projektion} & \textbf{gefilterte Rückprpjektion} \endhead
                \includegraphics[width=0.3\textwidth, height=0.2\textheight]{pic/Einzelfenster_Bilder/unbekannte_Quelle/unbek1_einf_prj.png}
                & 
                \includegraphics[width=.3\textwidth, height=0.2\textheight]{pic/Einzelfenster_Bilder/unbekannte_Quelle/unbek1gef_prj.png}\\
                \includegraphics[width=0.3\textwidth, height=0.2\textheight]{pic/Einzelfenster_Bilder/unbekannte_Quelle/unbek2einf_prj.png}
                & 
                \includegraphics[width=.3\textwidth, height=0.2\textheight]{pic/Einzelfenster_Bilder/unbekannte_Quelle/unbek2gef_prj.png}\\ 
                \includegraphics[width=0.3\textwidth, height=0.2\textheight]{pic/Einzelfenster_Bilder/unbekannte_Quelle/unbek3einf_prj.png}
                & 
                \includegraphics[width=.3\textwidth, height=0.2\textheight]{pic/Einzelfenster_Bilder/unbekannte_Quelle/unbek3gef_prj.png}\\               
                \includegraphics[width=0.3\textwidth, height=0.2\textheight]{pic/Einzelfenster_Bilder/unbekannte_Quelle/unbek4einf_prj.png}
                & 
                \includegraphics[width=.3\textwidth, height=0.2\textheight]{pic/Einzelfenster_Bilder/unbekannte_Quelle/unbek4gef_prj.png} \\
            \end{longtable}
            \captionof{figure}{Screenshots der Bildenstehung der gefilterten (rechts) und ungefilterten (links) Rückprojektion}
            \ \\
            \textbf{Untersuchung des Einflusses verschiedener Filter}\\
            \begin{center}
              \begin{longtable}{p{4.0cm}p{4.0cm}p{4.0cm}l}
                  \includegraphics[width=0.25\textwidth, height=0.15\textheight]{pic/Einzelfenster_Bilder/unbekannte_Quelle/unbek5_einf_prj.png}
                  ungefilterte Rückprojektion
                  & 
                  \includegraphics[width=.25\textwidth, height=0.15\textheight]{pic/Einzelfenster_Bilder/unbekannte_Quelle/unbek5_ramp.png}
                  Rampf-Filter
                  &
                  \includegraphics[width=0.25\textwidth, height=0.15\textheight]{pic/Einzelfenster_Bilder/unbekannte_Quelle/unbek5_hanning_weighted.png}
                  Hanning-weighted-Filter\\
                  \includegraphics[width=.25\textwidth, height=0.15\textheight]{pic/Einzelfenster_Bilder/unbekannte_Quelle/unbek5_middle.png} 
                  Middle-Filter
                  &
                  \includegraphics[width=0.25\textwidth, height=0.15\textheight]{pic/Einzelfenster_Bilder/unbekannte_Quelle/unbek5_rausch3.png}
                  Rauschfilter bei Dimension 3
                  & 
                  \includegraphics[width=.25\textwidth, height=0.15\textheight]{pic/Einzelfenster_Bilder/unbekannte_Quelle/unbek5_rausch13.png}               
                  Rauschfilter bei Dimension 13\\
                  \includegraphics[width=0.25\textwidth, height=0.15\textheight]{pic/Einzelfenster_Bilder/unbekannte_Quelle/unbek5_rausch25.png}
                  Rauschfilter bei Dimension 25
                  &
                  \includegraphics[width=.25\textwidth, height=0.15\textheight]{pic/Einzelfenster_Bilder/unbekannte_Quelle/unbek5_rausch36.png} 
                  Rauschfilter bei Dimension 36
                  &
                  \includegraphics[width=.25\textwidth, height=0.15\textheight]{pic/Einzelfenster_Bilder/unbekannte_Quelle/unbek5_shepp-logan.png}
                  Shepp-Logan-Filter
                \end{longtable}
            \end{center}
            \vspace{5mm}
            Der Standardwert der Dimension ist 13.      
                   
        \subsubsection{Messung mit einer Punktquelle, Phantom an-/insotroper Dichteverteilung}
        
        \textbf{Qualitative Gegenüberstellung an-/isotroper Dichteverteilung}\\
 
              \begin{longtable}{p{3cm}p{3cm}p{3cm}p{3cm}c} 
                  \multicolumn{2}{c}{\textbf{anisotrope Dichteverteilung}} & \multicolumn{2}{c}{\textbf{isotrope Dichteverteilung}}\endhead
                  gefiltert & ungefiltert & gefiltert & ungefiltert \endfoot
                  \includegraphics[width=.2\textwidth, height=0.125\textheight]{pic/Einzelfenster_Bilder/inhomogene_Messung/inhomo1einf_prj.png}
                  & 
                  \includegraphics[width=.2\textwidth, height=0.125\textheight]{pic/Einzelfenster_Bilder/inhomogene_Messung/inhomo1gef_prj.png}
                  &
                  \includegraphics[width=.2\textwidth, height=0.125\textheight]{pic/Einzelfenster_Bilder/isotrope_Messung/iso1einf_rueckprj.png}
                  & 
                  \includegraphics[width=.2\textwidth, height=0.125\textheight]{pic/Einzelfenster_Bilder/isotrope_Messung/iso1gef_prj.png}\\
                  \includegraphics[width=.2\textwidth, height=0.125\textheight]{pic/Einzelfenster_Bilder/inhomogene_Messung/inhomo2einf_rueckprj.png}
                  & 
                  \includegraphics[width=.2\textwidth, height=0.125\textheight]{pic/Einzelfenster_Bilder/inhomogene_Messung/inhomo2gef_prj.png}
                  &
                  \includegraphics[width=.2\textwidth, height=0.125\textheight]{pic/Einzelfenster_Bilder/isotrope_Messung/iso2einf_rueckprj.png}
                  & 
                  \includegraphics[width=.2\textwidth, height=0.125\textheight]{pic/Einzelfenster_Bilder/isotrope_Messung/iso2gef_prj.png}\\
                  \includegraphics[width=.2\textwidth, height=0.125\textheight]{pic/Einzelfenster_Bilder/inhomogene_Messung/inhomo3einf_rueckprj.png}
                  & 
                  \includegraphics[width=.2\textwidth, height=0.125\textheight]{pic/Einzelfenster_Bilder/inhomogene_Messung/inhomo3gef_prj.png}
                  &
                  \includegraphics[width=.2\textwidth, height=0.125\textheight]{pic/Einzelfenster_Bilder/isotrope_Messung/iso3einf_rueckprj.png}
                  & 
                  \includegraphics[width=.2\textwidth, height=0.125\textheight]{pic/Einzelfenster_Bilder/isotrope_Messung/iso3gef_prj.png}\\
                  \includegraphics[width=.2\textwidth, height=0.125\textheight]{pic/Einzelfenster_Bilder/inhomogene_Messung/inhomo4einf_rueckprj.png}
                  & 
                  \includegraphics[width=.2\textwidth, height=0.125\textheight]{pic/Einzelfenster_Bilder/inhomogene_Messung/inhomo4gef_prj.png}
                  &
                  \includegraphics[width=.2\textwidth, height=0.125\textheight]{pic/Einzelfenster_Bilder/isotrope_Messung/iso4einf_prj.png}
                  & 
                  \includegraphics[width=.2\textwidth, height=0.125\textheight]{pic/Einzelfenster_Bilder/isotrope_Messung/iso4gef_prj.png}\\
                  \includegraphics[width=.2\textwidth, height=0.125\textheight]{pic/Einzelfenster_Bilder/inhomogene_Messung/inhomo5einf_rueckprj.png}
                  & 
                  \includegraphics[width=.2\textwidth, height=0.125\textheight]{pic/Einzelfenster_Bilder/inhomogene_Messung/inhomo5gef_prj.png}
                  &
                  \includegraphics[width=.2\textwidth, height=0.125\textheight]{pic/Einzelfenster_Bilder/isotrope_Messung/iso6einf_prj.png}
                  & 
                  \includegraphics[width=.2\textwidth, height=0.125\textheight]{pic/Einzelfenster_Bilder/isotrope_Messung/iso6gef_prj.png}\\
                  \multicolumn{2}{c}{\includegraphics[width=.2\textwidth, height=0.25\textheight]{pic/Einzelfenster_Bilder/inhomogene_Messung/inhomo5sino.png}}
                  &
                  \multicolumn{2}{c}{\includegraphics[width=.2\textwidth, height=0.25\textheight]{pic/Einzelfenster_Bilder/isotrope_Messung/iso6sino.png}}
           \end{longtable}
        
        \textbf{Gegenüberstellung der registrierten Ereigniszahlen und Ermittlung einer Korrekturfunktion}\\
        
        \centering \includegraphics[width=.72\textwidth, height=0.225\textheight]{pic/isotropie.png}
        \captionof{figure}{Plot der registrierten Ereigniszahlen}
        \centering \includegraphics[width=.75\textwidth, height=0.25\textheight]{pic/KorrekturderAnisotropie.png}
        \captionof{figure}{registrierte Ereignisse (gelb), korrigierte Ereigniszahlen (schwarz) und Korrekturfunktion (blau)}
        \flushleft
        Die Korrekturfunktion, die man sich anhand der gelben Kurve ausdenken könnte, ist eine Gauß-Funktion, die ihr Maximum gerade im Minimum der erfassten Daten hat.
        Sie sieht folgendermaßen aus:
        \begin{equation*}
            K(\theta) = \frac{A}{\sqrt{2\pi}\sigma} e^{-\frac{1}{2}\left(\frac{\theta - \Delta \theta}{\sigma}\right)^2} \text{ mit folgenden Parametern } \sigma=\unit[14]{°}, A=\unit[1300]{°} \text{ und } \Delta \theta = \unit[98]{°}
        \end{equation*}
        
        
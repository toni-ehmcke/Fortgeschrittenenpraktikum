\section{Zusammenfassung}
In diesem Versuch waren vier Aufgaben zu erledigen. Zunächst sollte man sich damit beschäftigen, wie ein PET-Detektor theoretisch arbeitet. Dies
geschah anhand eines Rechenbeispiels, indem gezeigt und erklärt wird, wie ein Computer aus den gemessenen Signalen anhand von Matrizen eine Projektion
erzeugt. Dies geschah am Beispiel zunächst für die einfache und danach für die Bestimmung der gefilterten Rückprojektion. In Abbildung 4 sind die drei
signifikanten Schritte zu sehen. \\
Auf diesen theoretischen Aspekt folgt die Detektorkalibrierung. Diese erfolgt in drei Schritten. Der erste Schritt ist die Vermessung einer Quelle 
mit bekannter Aktivität genau mittig zwischen den beiden Detektoren. Daraus erhält man drei Datensätze, die für die Kalibrierung wichtig sind. Dies ist zum
einen das Zeitspektrum, aus welchem man das Koinzidenzzeitspektrum und die Koinzidenzauflösungszeit ermitteln, damit für die Rekonstruktion der Messdaten nur 
die zeitlich korrelierten Ereignisse verwendet werden. Für Detektor A und B erhält man jeweils Energiespektren, die man dazu verwendet, weitere Ereignisse 
herauszufiltern, die beispielsweise zu oft gestreut wurden oder schlichtweg aus der Umgebungsstrahlung oder anderer Zerfallsarten der Quelle herrühren.
Man erhält weiterhin ein Schwerpunktsdiagramm, auf dem man die Kristallstruktur des Detektors und damit die 8x8-Matrix erkennt. 
Danach finden Messungen statt, deren Quellposition direkt an beiden Detektoren ist. So ermittelt man den Detektorabstand aus der Differenz der gemessenen
Zeitunterschiede und kann erkennen, dass an dem gegenüberliegen Detektor jeweils die noch ein Peak mit geringeren Energien erscheint, der eventuell durch 
Streuprozesse und Bremsstrahlung erzeugt wurde. Die Grafiken, die für die Messungen stehen, an denen die Münze am Detektor liegt. Hier kann man vor allem auch 
den Prozess erkennen, der neben dem $\beta+$-Zerfall stattfindet. Über die erhaltenen Schwerpunktsdiagramme, die der Messplatz gleich mit erstellt und die eine Verteilung der Ereignisse über die Detektorfläche darstellen,
kann man weitere Eigenschaften der Detektoren bestimmten und den Abstand der beiden Detektoren voneinander zu bestimmen, sowie ein Gefühl für die Arbeitsweise
und die Messdaten der Detektoren zu erhalten. ($\varepsilon_A = \unit[(46\pm 3)]{\%},\ \varepsilon_B = \unit[(54\pm 3)]{\%}, \ L = (46,5\pm4,3)\unit{cm}$)
Dabei sind die Daten, welche aus Fits eines Datenanalysetools extrahiert wurden, alle mit Fehlern behaftet. Die Software selbst berechnet die Fehler dabei so, dass
sie möglichst klein sind und gegenüber der Größe auch tatsächlich vernachlässigbar scheinen. Nach einigen Tests von verschiedenen Funktionen in der Software QtiPlot
kann davon ausgegangen werden, dass die Fit-Funktionen solcher Software wesentlich fehlerbehafteter sind als hier für die Berechnung angenommen wurde.\\
Nach der Kalibrierung erfolgt der Hauptversuch: Es wird eine unbekannte Quellverteilung vermessen. Später wird die Wirkung unterschiedlicher Filter auf diese Messung untersucht
Es zeigt sich, dass die gegebenen Filter unterschiedliche Qualitäten haben. So zeigen sich Unterschiede in puncto Schärfe, Kontrast und Rauscheffekte, sowohl über das Bild als auch direkt an den Quellen.
Von Interesse ist hier auch das Sinogramm, welches bei mehreren Quellen auch mehrere Streifen zeigt. 
Danach schaut man sich die Unterschiede von isotroper und anisotropen Abschirmungen an. Man erfährt, dass Anisotropien stets gespiegelt aufgenommen werden, jedoch durch eine 
Korrekturfunktion, die es günstig zu erraten gilt, ausgeglichen werden können. 
In diesem Protokoll wurde als Korrektur eine Gauß-Funktion angewendet. ---
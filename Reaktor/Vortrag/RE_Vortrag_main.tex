\documentclass[10pt]{beamer}
\usetheme{Dresden}
\setbeamertemplate{navigation symbols}{} % remove navigation bar in the lower right corner


%\documentclass[ddcfooter]{tudbeamer}
\usepackage[utf8]{inputenc}
\usepackage[german]{babel}

\newcommand{\vtiny}{\fontsize{6pt}{30pt}\selectfont} % write source in the lower right corner

\usepackage{amsmath}
\usepackage{amsfonts}
\usepackage{amssymb}
\usepackage{graphicx}
\usepackage{caption}
\usepackage{tikz} 
\usepackage{ziffer}
\usepackage{units}
\usepackage{hyperref}
\usepackage{ wasysym } % use symbol for 'cent'

\usepackage{xmpmulti} %include gifs

\setbeamertemplate{caption}[numbered]

\title{Fortgeschrittenenpraktikum}
\subtitle{Kernreaktor} 
\author{Toni Ehmcke}
\institute{TU Dresden}
\date{\today}

\begin{document}

\maketitle

\section{Einführung}
	\begin{frame}{Warum wir nukleare Kräfte freisetzen wollen}
		\begin{itemize}
		\item Betrachte eine Masse von $m_U = 1\ \unit{g}$ des Uran-Nuklids $^{235}$U.\\
		\item Zahl der Atome in dieser Masse $N_{Atom} = \frac{m_U\cdot N_A}{M_{mol}} = 2,562 \cdot 10^{21}$
		\item Pro Kernspaltung freiwerdende Wärme $Q \approx 200\ \unit{MeV}$
		\item Summa summarum ergibt das eine maximale Energieabgabe von $Q_{ges}=N_{Atom}\cdot Q = 0,997\ \unit{MWd}$ 
		\item Spalten von $m_U = 1\ \unit{g}$ Uran-235 entspricht somit dem Verbrennen von $m_{BB} =4,39\ \unit{t}$ Braunkohlebriketts 
		\end{itemize}
	\end{frame}

\section{Ausbildungskernreaktor AKR-2} 
	\subsection{Aufbau und Maßnahmen zur nuklearen Sicherheit}
		\begin{frame}{}
				\begin{figure}[ht]
					\centering
					\includegraphics[width=0.7\textwidth]{pic/akr2.png}\\
					\text{\tiny{Quelle: TUD Institut für Energietechnik. \textit{AKR-2 Bau und Inbetriebnahme}, Dresden. Juli 2005}}
				\end{figure}	
		\end{frame}
	
		\begin{frame}{AKR-2: Aufbau (Querschnitt)}
			\begin{figure}[ht]
				\centering
				\includegraphics[width=0.5\textwidth]{pic/Vertikalquerschnitt_Reaktor.pdf}\\
				\text{\tiny{Quelle: TUD Institut für Energietechnik. \textit{AKR-2 Bau und Inbetriebnahme}, Dresden. Juli 2005}}
			\end{figure}	
		\end{frame}
		
		\begin{frame}{AKR-2: Aufbau (Azimutalebene)}
				\begin{figure}[ht]
					\centering
					\includegraphics[width=0.5\textwidth]{pic/Horizontalquerschnitt_Reaktor}\\
					\text{\tiny{Quelle: TUD Institut für Energietechnik. \textit{AKR-2 Bau und Inbetriebnahme}, Dresden. Juli 2005}}
				\end{figure}	
		\end{frame}
		
		\begin{frame}{AKR-2: Maßnahmen zur nuklearen Sicherheit}
				\begin{itemize}
				\item Unterdruck im Reaktortank\\
				\item Paraffin und Barytbeton für biologische Abschirmung
				\item Spaltzone in zwei Hälften geteilt
				\item mehrfach redundantes RESA-System, welches auslöst falls
					\begin{itemize}
						\item $P_{Reaktor} \notin [0,25;2,4]\ \unit{W}$
						\item Reaktorperiode $T_s < 10\ \unit{s}$ bzw. $T_2 < 7\ \unit{s}$
						\item Temperaturmessung $T < 18\ \unit{^\circ C}$
						\item Druckmessung $p > p_{max}$
						\item Fehlermeldung oder Ausfall im System
					\end{itemize}
				\end{itemize}		
		\end{frame}
		
		\begin{frame}{Dosimetrische Messungen}
		Gemessen wurde die mit Strahlungsart gewichtete Dosisleistung an verschie\-denen Orten und bei verschiedenen Leistungen im kritischen Reaktorzustand der Neutronen- und $\gamma$-Strahlung:\\
			\begin{table}
				\begin{tabular}{l|c|c|c|c}
			                \textbf{Ort} & \multicolumn{2}{c|}{\textbf{bei 1\unit{W}}} & \multicolumn{2}{c}{\textbf{bei 2\unit{W}}} \\
			                \hline       & $\dot D_\gamma$ [\textit{$\frac{\mu Sv}{h}$}] & $\dot D_n$ [\textit{$\frac{\mu Sv}{h}$}]& $\dot D_\gamma$ [\textit{$\frac{\mu Sv}{h}$}]& $\dot D_n$ [\textit{$\frac{\mu Sv}{h}$}]\\
			                \hline   Reaktortankwand ($\approx \unit[0]{m}$) & 12 & 2,5 & 27,5 & 4,3 \\
			                Operatortisch ($\approx \unit[3]{m}$) & 1,4 & 0,2 & 2,5 & 0,6 \\
			                Ecke ($\approx \unit[6]{m}$) & 0,6 & 0,14 & 0,5 & 0,13 
			 	\end{tabular}
			\end{table}
		\end{frame}
	
\section{Theoretische Grundlagen zum Kernreaktor}
	\subsection{Kernspaltung}
		\begin{frame}{Wirkungsquerschnitt für Neutroneneinfang}
			\begin{columns}
				\begin{column}[c]{0.6\linewidth}
					\begin{itemize}
						\item Spaltbarriere von $^{238}$U: $E_B^{238} = 5,7\ \unit{MeV}$
						\item Änderung der Bindungsenergie $\Delta B^{238} = 4,9\ \unit{MeV} < E_B^{238}$ \\
						$\rightarrow$ \glqq schnelle\grqq\ Neutronen notwendig
						\item Spaltbarriere von $^{235}$U: $E_B^{235} = 6,2\ \unit{MeV}$
						\item Änderung der Bindungsenergie $\Delta B^{235} = 6,5\ \unit{MeV} > E_B^{238}$\\ 
						$\rightarrow$ thermische Neutronen genügen
					\end{itemize}
				\end{column}
				
				\begin{column}[c]{0.4\linewidth}
					\begin{figure}[ht]
						\centering
						\includegraphics[width=\linewidth]{pic/Wirkungsquerschnitt_Uran}\\
						\tiny{Quelle: A. Ganczarczyk: \textit{Physikalische Grundlagen der Energieumwandlung}. \texttt{https://www.uni-due.de}. Duisburg, 01/2006, zuletzt geöffnet: 10.01.2016}
					\end{figure}
				\end{column}
			\end{columns}	
		\end{frame}
		
		\subsection{Reaktorkinetische Grundlagen}
				\begin{frame}{Reaktorkinetische Grundbegriffe}
					\begin{itemize}
						\item Entscheidend für den Zustand eines Reaktors ist die Bilanz der zwischen den Spaltprozessen entstandenen und vernichteten Neutronen
						\item Definiere den \textbf{Multiplikationsfaktor}
							\begin{equation*}
								k := \frac{N(t + l)}{N(t)} = \frac{N_{erzeugt}}{N_{absorbiert} + N_{leck}}
							\end{equation*}		
						\item sowie die relative Abweichung vom kritischen Zustand ($k=1$), die \textbf{Reaktivität}:
							\begin{equation*}
								\rho := \frac{k-1}{k}
							\end{equation*}
						\item Als Messgröße wird die \textbf{Reaktorperiode} $T_s$, die der Zeit entspricht, in der die Neutronendichte um den $e$-ten Teil zugenommen hat, genutzt.
					\end{itemize}	
		\end{frame}
		
		\begin{frame}{Neutroneninduzierte Spaltung von $^{235}$U und prompte Neutronen}
			\begin{figure}[ht]
				\centering
				\includegraphics[width=0.5\textwidth]{pic/Kernspaltung}\\
				\tiny{Quelle: \url{https://de.wikipedia.org/wiki/Kernspaltung}, zuletzt geöffnet: 28.01.16}
			\end{figure}
			\textbf{Problem}: Ein Reaktionszyklus findet auf einer Zeitskala von $l_p = 10^{-4}\ \unit{s}$\\
			$\rightarrow$ Kettenreaktion nicht kontrollierbar
		\end{frame}
		
		\begin{frame}{Wie man die Reaktion dennoch Steuern kann: Verzögerte Neutronen}
			\begin{itemize}
				\item Tochterkerne zerfallen weiter, wobei höher angeregte Tochterkerne höherer Generation entstehen können
				\item Abregung durch Emission von \textbf{verzögerten Neutronen}
				\item Lebensdauer (Zeitskala) dieser Neutronen wird durch HWZ der Mutterkerne im Mittel zu $l_v = 13\ \unit{s}$
				\item Diese machen einen Anteil von $\beta = 0,64\ \unit{\%}$ aus
				\item Effektive Lebensdauer $l_{eff} = \beta \cdot l_v + (1-\beta) \cdot l_p = 0.083\ \unit{s}$\\
				$\rightarrow$ \textbf{Reaktion wird erst durch verzögerte Neutronen kontrollierbar}
			\end{itemize}	
		\end{frame}
		
		\begin{frame}{Zeitverhalten eines verzögert überkritischen Reaktors}
				\begin{columns}
						\begin{column}[c]{0.6\linewidth}
							\begin{itemize}
								\item Lösung der \textbf{reaktorkinetischen Gleichungen} für eine zur Zeit $t=0$ zugeführte Reaktivität $\rho(t) = \rho_0\cdot \Theta(t)$:
								\begin{equation*}
								 \footnotesize{
									N(t)= N(0) \left[ c_1 e^{\frac{\lambda \cdot \rho_0}{\beta - \rho_0}\cdot t} - c_2 e^{-k\frac{\beta - \rho_0}{l_p} \cdot t} \right]
									}
								\end{equation*}
								\item Falls $0 < \rho_0 < \beta$, klingt der zweite Summand schnell ab und mit $T_s = (\beta - \rho_0)/\lambda\cdot \rho_0$ erhält man die im Experiment genutzte Beziehung $P(t) \propto N(t)\propto e^{t/T_s} = 2^{t/T_2}$
							\end{itemize}
						\end{column}
									
						\begin{column}[c]{0.4\linewidth}
							\begin{figure}[ht]
								\centering
								\includegraphics[width=\textwidth]{pic/Leistungsaenderung.pdf}\\
								\tiny{Quelle: Technische Universität Dresden,  Institut für Energietechnik Ausbildungskernreaktor: \textit{Reaktorpraktikum Versuch "Reaktorstart"}. Dresden, 05/2011}
							\end{figure}
						\end{column}
	     		\end{columns}	
		\end{frame}
		
		\begin{frame}{Wie man Reaktivität misst: Die Inhour-Gleichung}
					\begin{itemize}
						\item setzt man bestimmte Lösungen der reaktorkinetischen Gleichungen in die DGl ein, erhält man die Inhour-Gleichung:
							\begin{equation*}
								\rho\prime = \frac{\rho}{\beta} = \frac{l_p}{k\cdot\beta \cdot T_s}+ \sum_{i=1}^{6} \frac{a_i}{1 + \lambda_i\cdot T_s}
							\end{equation*}
						\item diese liefert bei Messung der Verdopplungszeit (oder stabilen Reaktorperiode) die zugeführte Reaktivität nach Verlassen des kritischen Zustandes
						\item als Einheit von $\rho\prime$ definiert man $[\rho\prime] = 1\ \unit{\$} = 100\ \unit{\cent}$
						\item solange $\rho\prime < 1$, befindet sich der Reaktor nicht im prompt überkritischen Zustand, welcher dringendst zu vermeiden ist 
					\end{itemize}	
		\end{frame}

\section{Steuerstabkalibrierung}
	\subsection{Differentielle und integrale Steuerstabkennlinie}
	
		\begin{frame}{Neutronenflussdichte}
			\begin{columns}
				\begin{column}[c]{0.6\linewidth}
					\begin{itemize}
						\item Ziel: Ermitteln des Einflusses der Steuer- und Regeleinrichtungen auf die Neutronenbilanz (Reaktivität)
						\item Annahme: Da Spaltzone zylindersymmetrisch, betrachte die Neutronenflussdichte $\Phi(z)$ ebenfalls als axialsymmetrisch
						\item Ein möglicher analytischer Ausdruck, um die Verteilung einer Spaltzone der Länge $H$ zu modellieren ist:
							\begin{equation*}
								\Phi(z) = \Phi_{max} \cdot \sin\left(\frac{\pi\cdot z}{H}\right)
							\end{equation*}
					\end{itemize}	
				\end{column}
												
				\begin{column}[c]{0.4\linewidth}
					\begin{figure}[ht]
						\centering
						\includegraphics[width=\textwidth]{pic/NeutrFlDichte}\\
						\tiny{Quelle: Technische Universität Dresden,  Institut für Energietechnik Ausbildungskernreaktor: \textit{Reaktorpraktikum Versuch \glqq Steuerstabkalibrierung\grqq}. Dresden, 05/2011}
					\end{figure}
				\end{column}
     		\end{columns}
 		\end{frame}
 		
 		\begin{frame}{Steuerstabkennlinien}
 			\begin{itemize}
 			 	\item Verschiebung des Steuerstabes um $dz$ ergibt die \textbf{differentielle Steuerstabkennlinie}:
 			 		\begin{equation*}
 			 			\mathrm{d}\rho \propto - \sigma_{abs} \cdot \Phi^2(z) \cdot \mathrm{d}z
 			 		\end{equation*}
 			 	\item Integration über eine endliche Länge z liefert die \textbf{integrale Steuerstabkennlinie}:
 			 		$$\frac{\rho(z)}{\rho_{max}} = \frac{z}{H} - \frac{1}{2\pi} \sin{\left(\frac{2\pi z}{H}\right)}$$
 			\end{itemize}
 			\begin{figure}
 			 	\includegraphics[width=0.6\linewidth]{pic/sskl}\\
 			 	\tiny{Quelle: Technische Universität Dresden,  Institut für Energietechnik Ausbildungskernreaktor: \textit{Reaktorpraktikum Versuch \glqq Steuerstabkalibrierung\grqq}. Dresden, 05/2011}
 			 \end{figure}
 		\end{frame}
 		
 		\begin{frame}{Messung: Kompensationsverfahren}
 			\begin{table}
 				\begin{tabular}{c|c|c|c|c|c}
 				    	Zustand	&	$z_1\ [\unit{digit}]$	&	$z_2\ [\unit{digit}]$	&	$z_3\ [\unit{digit}]$	&	$T_2\ [\unit{s}]$	&	$T_s\ [\unit{s}]$\\
 				    	\hline
 				    	Kritisch	&	0		&	4.000	&	2.924	&	-	&	-	\\
 				    	ÜK			&	827		&	4.000	&	2.924	&	127	&	183\\
 				    	Kritisch	&	827		&	3.371	&	2.924	&	-	&	-\\
 				    	ÜK			&	1.609	&	3.371	&	2.924	&	97	&	140\\
 				    	Kritisch	&	1.609	&	2.626	&	2.924	&	-	&	-\\
 				    	ÜK			&	2.389	&	2.626	&	2.924	&	72	&	104\\
 				    	Kritisch	&	2.389	&	1.926	&	2.924	&	-	&	-\\
 				    	ÜK			&	3.245	&	1.926	&	2.924	&	96	&	139\\
 				    	Kritisch	&	3.245	&	804		&	2.924	&	-	&	-\\
 				    	ÜK			&	4.000	&	804		&	2.924	&	121	&	175\\
 				    	Kritisch	&	4.000	&	0		&	2.802	&	-	&	-	
 				    	\end{tabular}
 			\end{table}	
 		\end{frame}
 		
 		\begin{frame}{Messergebnis: differentielle Steuerstabkennlinie}
 			\begin{figure}[hp]
 		 		\centering
 				\scalebox{0.4}{
 				\begin{tikzpicture}{0pt}{0pt}{436pt}{281pt}
	\clip(0pt,562pt) -- (656.452pt,562pt) -- (656.452pt,138.919pt) -- (0pt,138.919pt) -- (0pt,562pt);
\begin{scope}
	\clip(60.225pt,504.786pt) -- (595.475pt,504.786pt) -- (595.475pt,197.639pt) -- (60.225pt,197.639pt) -- (60.225pt,504.786pt);
	\color[rgb]{0.627451,0.627451,0.643137}
	\draw[line width=0.4pt, dash pattern=on 0.0096cm off 0.032cm, dash phase=0pt, line join=bevel, line cap=rect](86.5734pt,504.786pt) -- (86.5734pt,198.392pt);
	\color[rgb]{0.627451,0.627451,0.643137}
	\draw[line width=0.4pt, dash pattern=on 0.0096cm off 0.032cm, dash phase=0pt, line join=bevel, line cap=rect](112.169pt,504.786pt) -- (112.169pt,198.392pt);
	\draw[line width=0.4pt, dash pattern=on 0.0096cm off 0.032cm, dash phase=0pt, line join=bevel, line cap=rect](138.517pt,504.786pt) -- (138.517pt,198.392pt);
	\draw[line width=0.4pt, dash pattern=on 0.0096cm off 0.032cm, dash phase=0pt, line join=bevel, line cap=rect](164.866pt,504.786pt) -- (164.866pt,198.392pt);
	\draw[line width=0.4pt, dash pattern=on 0.0096cm off 0.032cm, dash phase=0pt, line join=bevel, line cap=rect](216.81pt,504.786pt) -- (216.81pt,198.392pt);
	\draw[line width=0.4pt, dash pattern=on 0.0096cm off 0.032cm, dash phase=0pt, line join=bevel, line cap=rect](243.158pt,504.786pt) -- (243.158pt,198.392pt);
	\draw[line width=0.4pt, dash pattern=on 0.0096cm off 0.032cm, dash phase=0pt, line join=bevel, line cap=rect](268.754pt,504.786pt) -- (268.754pt,198.392pt);
	\draw[line width=0.4pt, dash pattern=on 0.0096cm off 0.032cm, dash phase=0pt, line join=bevel, line cap=rect](295.102pt,504.786pt) -- (295.102pt,198.392pt);
	\draw[line width=0.4pt, dash pattern=on 0.0096cm off 0.032cm, dash phase=0pt, line join=bevel, line cap=rect](347.799pt,504.786pt) -- (347.799pt,198.392pt);
	\draw[line width=0.4pt, dash pattern=on 0.0096cm off 0.032cm, dash phase=0pt, line join=bevel, line cap=rect](373.395pt,504.786pt) -- (373.395pt,198.392pt);
	\draw[line width=0.4pt, dash pattern=on 0.0096cm off 0.032cm, dash phase=0pt, line join=bevel, line cap=rect](399.743pt,504.786pt) -- (399.743pt,198.392pt);
	\draw[line width=0.4pt, dash pattern=on 0.0096cm off 0.032cm, dash phase=0pt, line join=bevel, line cap=rect](426.092pt,504.786pt) -- (426.092pt,198.392pt);
	\draw[line width=0.4pt, dash pattern=on 0.0096cm off 0.032cm, dash phase=0pt, line join=bevel, line cap=rect](478.036pt,504.786pt) -- (478.036pt,198.392pt);
	\draw[line width=0.4pt, dash pattern=on 0.0096cm off 0.032cm, dash phase=0pt, line join=bevel, line cap=rect](504.384pt,504.786pt) -- (504.384pt,198.392pt);
	\draw[line width=0.4pt, dash pattern=on 0.0096cm off 0.032cm, dash phase=0pt, line join=bevel, line cap=rect](529.98pt,504.786pt) -- (529.98pt,198.392pt);
	\draw[line width=0.4pt, dash pattern=on 0.0096cm off 0.032cm, dash phase=0pt, line join=bevel, line cap=rect](556.328pt,504.786pt) -- (556.328pt,198.392pt);
	\draw[line width=0.4pt, dash pattern=on 0.0096cm off 0.032cm, dash phase=0pt, line join=bevel, line cap=rect](60.225pt,212.695pt) -- (594.722pt,212.695pt);
	\draw[line width=0.4pt, dash pattern=on 0.0096cm off 0.032cm, dash phase=0pt, line join=bevel, line cap=rect](60.225pt,243.56pt) -- (594.722pt,243.56pt);
	\draw[line width=0.4pt, dash pattern=on 0.0096cm off 0.032cm, dash phase=0pt, line join=bevel, line cap=rect](60.225pt,274.426pt) -- (594.722pt,274.426pt);
	\draw[line width=0.4pt, dash pattern=on 0.0096cm off 0.032cm, dash phase=0pt, line join=bevel, line cap=rect](60.225pt,305.291pt) -- (594.722pt,305.291pt);
	\draw[line width=0.4pt, dash pattern=on 0.0096cm off 0.032cm, dash phase=0pt, line join=bevel, line cap=rect](60.225pt,336.156pt) -- (594.722pt,336.156pt);
	\draw[line width=0.4pt, dash pattern=on 0.0096cm off 0.032cm, dash phase=0pt, line join=bevel, line cap=rect](60.225pt,366.269pt) -- (594.722pt,366.269pt);
	\draw[line width=0.4pt, dash pattern=on 0.0096cm off 0.032cm, dash phase=0pt, line join=bevel, line cap=rect](60.225pt,397.134pt) -- (594.722pt,397.134pt);
	\draw[line width=0.4pt, dash pattern=on 0.0096cm off 0.032cm, dash phase=0pt, line join=bevel, line cap=rect](60.225pt,427.999pt) -- (594.722pt,427.999pt);
	\draw[line width=0.4pt, dash pattern=on 0.0096cm off 0.032cm, dash phase=0pt, line join=bevel, line cap=rect](60.225pt,458.865pt) -- (594.722pt,458.865pt);
	\draw[line width=0.4pt, dash pattern=on 0.0096cm off 0.032cm, dash phase=0pt, line join=bevel, line cap=rect](60.225pt,489.73pt) -- (594.722pt,489.73pt);
	\draw[line width=0.4pt, dash pattern=on 0.0096cm off 0.032cm, dash phase=0pt, line join=bevel, line cap=rect](60.225pt,228.504pt) -- (594.722pt,228.504pt);
	\draw[line width=0.4pt, dash pattern=on 0.0096cm off 0.032cm, dash phase=0pt, line join=bevel, line cap=rect](60.225pt,289.482pt) -- (594.722pt,289.482pt);
	\draw[line width=0.4pt, dash pattern=on 0.0096cm off 0.032cm, dash phase=0pt, line join=bevel, line cap=rect](60.225pt,351.213pt) -- (594.722pt,351.213pt);
	\draw[line width=0.4pt, dash pattern=on 0.0096cm off 0.032cm, dash phase=0pt, line join=bevel, line cap=rect](60.225pt,412.943pt) -- (594.722pt,412.943pt);
	\draw[line width=0.4pt, dash pattern=on 0.0096cm off 0.032cm, dash phase=0pt, line join=bevel, line cap=rect](60.225pt,473.921pt) -- (594.722pt,473.921pt);
	\color[rgb]{0,0,1}
	\draw[line width=0.5pt, line join=bevel, line cap=rect](190.462pt,504.786pt) -- (190.462pt,198.392pt);
	\draw[line width=0.5pt, line join=bevel, line cap=rect](321.451pt,504.786pt) -- (321.451pt,198.392pt);
	\draw[line width=0.5pt, line join=bevel, line cap=rect](451.688pt,504.786pt) -- (451.688pt,198.392pt);
	\draw[line width=0.5pt, line join=bevel, line cap=rect](582.677pt,504.786pt) -- (582.677pt,198.392pt);
	\draw[line width=0.5pt, line join=bevel, line cap=rect](60.225pt,259.369pt) -- (594.722pt,259.369pt);
	\draw[line width=0.5pt, line join=bevel, line cap=rect](60.225pt,320.347pt) -- (594.722pt,320.347pt);
	\draw[line width=0.5pt, line join=bevel, line cap=rect](60.225pt,382.078pt) -- (594.722pt,382.078pt);
	\draw[line width=0.5pt, line join=bevel, line cap=rect](60.225pt,443.056pt) -- (594.722pt,443.056pt);
	\color[rgb]{0,0,0}
	\draw[line width=1pt, dash pattern=on 0.12cm off 0.08cm, dash phase=0pt, line join=miter, line cap=rect](60.225pt,197.639pt) -- (112.706pt,390.22pt) -- (238.424pt,431.378pt) -- (357.354pt,492.5pt) -- (451.675pt,429.535pt) -- (541.362pt,382.542pt) -- (582.014pt,197.639pt);
	\draw[line width=1pt, line join=miter, line cap=rect](112.922pt,392.617pt) -- (112.922pt,388.1pt);
	\draw[line width=1pt, line join=miter, line cap=rect](110.663pt,390.359pt) -- (115.18pt,390.359pt);
	\draw[line width=1pt, line join=miter, line cap=rect](238.642pt,433.269pt) -- (238.642pt,428.752pt);
	\draw[line width=1pt, line join=miter, line cap=rect](236.383pt,431.011pt) -- (240.9pt,431.011pt);
	\draw[line width=1pt, line join=miter, line cap=rect](357.586pt,495pt) -- (357.586pt,490.483pt);
	\draw[line width=1pt, line join=miter, line cap=rect](355.327pt,492.741pt) -- (359.844pt,492.741pt);
	\draw[line width=1pt, line join=miter, line cap=rect](451.688pt,431.763pt) -- (451.688pt,427.247pt);
	\draw[line width=1pt, line join=miter, line cap=rect](449.429pt,429.505pt) -- (453.946pt,429.505pt);
	\draw[line width=1pt, line join=miter, line cap=rect](541.272pt,385.089pt) -- (541.272pt,380.572pt);
	\draw[line width=1pt, line join=miter, line cap=rect](539.014pt,382.831pt) -- (543.531pt,382.831pt);
	\color[rgb]{1,0,0}
	\draw[line width=1pt, line join=miter, line cap=rect](60.225pt,197.639pt) -- (114.207pt,382.542pt) -- (219.233pt,429.535pt) -- (321.192pt,492.5pt) -- (427.981pt,431.378pt) -- (533.138pt,390.22pt) -- (582.07pt,197.639pt);
	\draw[line width=1pt, line join=miter, line cap=rect](114.427pt,385.089pt) -- (114.427pt,380.572pt);
	\draw[line width=1pt, line join=miter, line cap=rect](112.169pt,382.831pt) -- (116.686pt,382.831pt);
	\draw[line width=1pt, line join=miter, line cap=rect](219.068pt,431.763pt) -- (219.068pt,427.247pt);
	\draw[line width=1pt, line join=miter, line cap=rect](216.81pt,429.505pt) -- (221.327pt,429.505pt);
	\draw[line width=1pt, line join=miter, line cap=rect](321.451pt,495pt) -- (321.451pt,490.483pt);
	\draw[line width=1pt, line join=miter, line cap=rect](319.192pt,492.741pt) -- (323.709pt,492.741pt);
	\draw[line width=1pt, line join=miter, line cap=rect](428.35pt,433.269pt) -- (428.35pt,428.752pt);
	\draw[line width=1pt, line join=miter, line cap=rect](426.092pt,431.011pt) -- (430.609pt,431.011pt);
	\draw[line width=1pt, line join=miter, line cap=rect](532.991pt,392.617pt) -- (532.991pt,388.1pt);
	\draw[line width=1pt, line join=miter, line cap=rect](530.733pt,390.359pt) -- (535.25pt,390.359pt);
	\draw[line width=1pt, line join=miter, line cap=rect](114.207pt,384.8pt) -- (114.207pt,385.613pt);
	\draw[line width=1pt, line join=miter, line cap=rect](111.196pt,385.613pt) -- (117.218pt,385.613pt);
	\draw[line width=1pt, line join=miter, line cap=rect](114.207pt,380.283pt) -- (114.207pt,379.47pt);
	\draw[line width=1pt, line join=miter, line cap=rect](111.196pt,379.47pt) -- (117.218pt,379.47pt);
	\draw[line width=1pt, line join=miter, line cap=rect](114.207pt,384.8pt) -- (114.207pt,380.283pt);
	\draw[line width=1pt, line join=miter, line cap=rect](219.233pt,431.794pt) -- (219.233pt,432.607pt);
	\draw[line width=1pt, line join=miter, line cap=rect](216.222pt,432.607pt) -- (222.245pt,432.607pt);
	\draw[line width=1pt, line join=miter, line cap=rect](219.233pt,427.277pt) -- (219.233pt,426.464pt);
	\draw[line width=1pt, line join=miter, line cap=rect](216.222pt,426.464pt) -- (222.245pt,426.464pt);
	\draw[line width=1pt, line join=miter, line cap=rect](219.233pt,431.794pt) -- (219.233pt,427.277pt);
	\draw[line width=1pt, line join=miter, line cap=rect](321.192pt,494.759pt) -- (321.192pt,495.572pt);
	\draw[line width=1pt, line join=miter, line cap=rect](318.181pt,495.572pt) -- (324.203pt,495.572pt);
	\draw[line width=1pt, line join=miter, line cap=rect](321.192pt,490.242pt) -- (321.192pt,489.429pt);
	\draw[line width=1pt, line join=miter, line cap=rect](318.181pt,489.429pt) -- (324.203pt,489.429pt);
	\draw[line width=1pt, line join=miter, line cap=rect](321.192pt,494.759pt) -- (321.192pt,490.242pt);
	\draw[line width=1pt, line join=miter, line cap=rect](427.981pt,433.636pt) -- (427.981pt,434.449pt);
	\draw[line width=1pt, line join=miter, line cap=rect](424.969pt,434.449pt) -- (430.992pt,434.449pt);
	\draw[line width=1pt, line join=miter, line cap=rect](427.981pt,429.12pt) -- (427.981pt,428.307pt);
	\draw[line width=1pt, line join=miter, line cap=rect](424.969pt,428.307pt) -- (430.992pt,428.307pt);
	\draw[line width=1pt, line join=miter, line cap=rect](427.981pt,433.636pt) -- (427.981pt,429.12pt);
	\draw[line width=1pt, line join=miter, line cap=rect](533.138pt,392.479pt) -- (533.138pt,393.292pt);
	\draw[line width=1pt, line join=miter, line cap=rect](530.126pt,393.292pt) -- (536.149pt,393.292pt);
	\draw[line width=1pt, line join=miter, line cap=rect](533.138pt,387.962pt) -- (533.138pt,387.149pt);
	\draw[line width=1pt, line join=miter, line cap=rect](530.126pt,387.149pt) -- (536.149pt,387.149pt);
	\draw[line width=1pt, line join=miter, line cap=rect](533.138pt,392.479pt) -- (533.138pt,387.962pt);
	\color[rgb]{0,0,0}
	\draw[line width=1pt, line join=miter, line cap=rect](112.706pt,392.479pt) -- (112.706pt,393.292pt);
	\draw[line width=1pt, line join=miter, line cap=rect](109.694pt,393.292pt) -- (115.717pt,393.292pt);
	\draw[line width=1pt, line join=miter, line cap=rect](112.706pt,387.962pt) -- (112.706pt,387.149pt);
	\draw[line width=1pt, line join=miter, line cap=rect](109.694pt,387.149pt) -- (115.717pt,387.149pt);
	\draw[line width=1pt, line join=miter, line cap=rect](112.706pt,392.479pt) -- (112.706pt,387.962pt);
	\draw[line width=1pt, line join=miter, line cap=rect](238.424pt,433.636pt) -- (238.424pt,434.449pt);
	\draw[line width=1pt, line join=miter, line cap=rect](235.413pt,434.449pt) -- (241.435pt,434.449pt);
	\draw[line width=1pt, line join=miter, line cap=rect](238.424pt,429.12pt) -- (238.424pt,428.307pt);
	\draw[line width=1pt, line join=miter, line cap=rect](235.413pt,428.307pt) -- (241.435pt,428.307pt);
	\draw[line width=1pt, line join=miter, line cap=rect](238.424pt,433.636pt) -- (238.424pt,429.12pt);
	\draw[line width=1pt, line join=miter, line cap=rect](357.354pt,494.759pt) -- (357.354pt,495.572pt);
	\draw[line width=1pt, line join=miter, line cap=rect](354.343pt,495.572pt) -- (360.365pt,495.572pt);
	\draw[line width=1pt, line join=miter, line cap=rect](357.354pt,490.242pt) -- (357.354pt,489.429pt);
	\draw[line width=1pt, line join=miter, line cap=rect](354.343pt,489.429pt) -- (360.365pt,489.429pt);
	\draw[line width=1pt, line join=miter, line cap=rect](357.354pt,494.759pt) -- (357.354pt,490.242pt);
	\draw[line width=1pt, line join=miter, line cap=rect](451.675pt,431.794pt) -- (451.675pt,432.607pt);
	\draw[line width=1pt, line join=miter, line cap=rect](448.664pt,432.607pt) -- (454.687pt,432.607pt);
	\draw[line width=1pt, line join=miter, line cap=rect](451.675pt,427.277pt) -- (451.675pt,426.464pt);
	\draw[line width=1pt, line join=miter, line cap=rect](448.664pt,426.464pt) -- (454.687pt,426.464pt);
	\draw[line width=1pt, line join=miter, line cap=rect](451.675pt,431.794pt) -- (451.675pt,427.277pt);
	\draw[line width=1pt, line join=miter, line cap=rect](541.362pt,384.8pt) -- (541.362pt,385.613pt);
	\draw[line width=1pt, line join=miter, line cap=rect](538.351pt,385.613pt) -- (544.373pt,385.613pt);
	\draw[line width=1pt, line join=miter, line cap=rect](541.362pt,380.283pt) -- (541.362pt,379.47pt);
	\draw[line width=1pt, line join=miter, line cap=rect](538.351pt,379.47pt) -- (544.373pt,379.47pt);
	\draw[line width=1pt, line join=miter, line cap=rect](541.362pt,384.8pt) -- (541.362pt,380.283pt);
\end{scope}
\begin{scope}
	\color[rgb]{0,0,0}
	\pgftext[center, base, at={\pgfpoint{328.22pt}{541.674pt}}]{\fontsize{15}{0}\selectfont{\textbf{Differentielle Steuerstabkennlinien}}}
	\color[rgb]{0,0,0}
	\pgftext[center, base, at={\pgfpoint{19.5731pt}{310.761pt}},rotate=90]{\fontsize{12}{0}\selectfont{$\mathrm{d} \rho / \mathrm{d}z\ [\unit{\$ /digit}]$}}
	\color[rgb]{0.129412,0.129412,0.129412}
	\pgftext[center, base, at={\pgfpoint{45.5981pt}{193.122pt}}]{\fontsize{12}{0}\selectfont{0}}
	\pgftext[center, base, at={\pgfpoint{37.5289pt}{254.853pt}}]{\fontsize{12}{0}\selectfont{0.02}}
	\pgftext[center, base, at={\pgfpoint{37.5289pt}{315.83pt}}]{\fontsize{12}{0}\selectfont{0.04}}
	\pgftext[center, base, at={\pgfpoint{37.5289pt}{377.561pt}}]{\fontsize{12}{0}\selectfont{0.06}}
	\pgftext[center, base, at={\pgfpoint{37.5289pt}{438.539pt}}]{\fontsize{12}{0}\selectfont{0.08}}
	\pgftext[center, base, at={\pgfpoint{41.6164pt}{500.269pt}}]{\fontsize{12}{0}\selectfont{0.1}}
	\color[rgb]{0,0,0}
	\draw[line width=1pt, line join=bevel, line cap=rect](60.225pt,212.695pt) -- (56.4609pt,212.695pt);
	\draw[line width=1pt, line join=bevel, line cap=rect](60.225pt,243.56pt) -- (56.4609pt,243.56pt);
	\draw[line width=1pt, line join=bevel, line cap=rect](60.225pt,274.426pt) -- (56.4609pt,274.426pt);
	\draw[line width=1pt, line join=bevel, line cap=rect](60.225pt,305.291pt) -- (56.4609pt,305.291pt);
	\draw[line width=1pt, line join=bevel, line cap=rect](60.225pt,336.156pt) -- (56.4609pt,336.156pt);
	\draw[line width=1pt, line join=bevel, line cap=rect](60.225pt,366.269pt) -- (56.4609pt,366.269pt);
	\draw[line width=1pt, line join=bevel, line cap=rect](60.225pt,397.134pt) -- (56.4609pt,397.134pt);
	\draw[line width=1pt, line join=bevel, line cap=rect](60.225pt,427.999pt) -- (56.4609pt,427.999pt);
	\draw[line width=1pt, line join=bevel, line cap=rect](60.225pt,458.865pt) -- (56.4609pt,458.865pt);
	\draw[line width=1pt, line join=bevel, line cap=rect](60.225pt,489.73pt) -- (56.4609pt,489.73pt);
	\draw[line width=1pt, line join=bevel, line cap=rect](60.225pt,228.504pt) -- (56.4609pt,228.504pt);
	\draw[line width=1pt, line join=bevel, line cap=rect](60.225pt,289.482pt) -- (56.4609pt,289.482pt);
	\draw[line width=1pt, line join=bevel, line cap=rect](60.225pt,351.213pt) -- (56.4609pt,351.213pt);
	\draw[line width=1pt, line join=bevel, line cap=rect](60.225pt,412.943pt) -- (56.4609pt,412.943pt);
	\draw[line width=1pt, line join=bevel, line cap=rect](60.225pt,473.921pt) -- (56.4609pt,473.921pt);
	\draw[line width=1pt, line join=bevel, line cap=rect](60.225pt,197.639pt) -- (53.4497pt,197.639pt);
	\draw[line width=1pt, line join=bevel, line cap=rect](60.225pt,259.369pt) -- (53.4497pt,259.369pt);
	\draw[line width=1pt, line join=bevel, line cap=rect](60.225pt,320.347pt) -- (53.4497pt,320.347pt);
	\draw[line width=1pt, line join=bevel, line cap=rect](60.225pt,382.078pt) -- (53.4497pt,382.078pt);
	\draw[line width=1pt, line join=bevel, line cap=rect](60.225pt,443.056pt) -- (53.4497pt,443.056pt);
	\draw[line width=1pt, line join=bevel, line cap=rect](60.225pt,504.786pt) -- (53.4497pt,504.786pt);
	\draw[line width=1pt, line join=bevel, line cap=rect](60.225pt,504.786pt) -- (60.225pt,197.639pt);
	\pgftext[center, base, at={\pgfpoint{645.913pt}{351.589pt}},rotate=90]{\fontsize{12}{0}\selectfont{}}
	\color[rgb]{0.129412,0.129412,0.129412}
	\pgftext[center, base, at={\pgfpoint{609.455pt}{193.122pt}}]{\fontsize{12}{0}\selectfont{0}}
	\pgftext[center, base, at={\pgfpoint{617.947pt}{254.853pt}}]{\fontsize{12}{0}\selectfont{0.02}}
	\pgftext[center, base, at={\pgfpoint{617.947pt}{315.83pt}}]{\fontsize{12}{0}\selectfont{0.04}}
	\pgftext[center, base, at={\pgfpoint{617.947pt}{377.561pt}}]{\fontsize{12}{0}\selectfont{0.06}}
	\pgftext[center, base, at={\pgfpoint{617.947pt}{438.539pt}}]{\fontsize{12}{0}\selectfont{0.08}}
	\pgftext[center, base, at={\pgfpoint{614.507pt}{500.269pt}}]{\fontsize{12}{0}\selectfont{0.1}}
	\color[rgb]{0,0,0}
	\draw[line width=1pt, line join=bevel, line cap=rect](595.475pt,212.695pt) -- (599.239pt,212.695pt);
	\draw[line width=1pt, line join=bevel, line cap=rect](595.475pt,243.56pt) -- (599.239pt,243.56pt);
	\draw[line width=1pt, line join=bevel, line cap=rect](595.475pt,274.426pt) -- (599.239pt,274.426pt);
	\draw[line width=1pt, line join=bevel, line cap=rect](595.475pt,305.291pt) -- (599.239pt,305.291pt);
	\draw[line width=1pt, line join=bevel, line cap=rect](595.475pt,336.156pt) -- (599.239pt,336.156pt);
	\draw[line width=1pt, line join=bevel, line cap=rect](595.475pt,366.269pt) -- (599.239pt,366.269pt);
	\draw[line width=1pt, line join=bevel, line cap=rect](595.475pt,397.134pt) -- (599.239pt,397.134pt);
	\draw[line width=1pt, line join=bevel, line cap=rect](595.475pt,427.999pt) -- (599.239pt,427.999pt);
	\draw[line width=1pt, line join=bevel, line cap=rect](595.475pt,458.865pt) -- (599.239pt,458.865pt);
	\draw[line width=1pt, line join=bevel, line cap=rect](595.475pt,489.73pt) -- (599.239pt,489.73pt);
	\draw[line width=1pt, line join=bevel, line cap=rect](595.475pt,228.504pt) -- (599.239pt,228.504pt);
	\draw[line width=1pt, line join=bevel, line cap=rect](595.475pt,289.482pt) -- (599.239pt,289.482pt);
	\draw[line width=1pt, line join=bevel, line cap=rect](595.475pt,351.213pt) -- (599.239pt,351.213pt);
	\draw[line width=1pt, line join=bevel, line cap=rect](595.475pt,412.943pt) -- (599.239pt,412.943pt);
	\draw[line width=1pt, line join=bevel, line cap=rect](595.475pt,473.921pt) -- (599.239pt,473.921pt);
	\draw[line width=1pt, line join=bevel, line cap=rect](595.475pt,197.639pt) -- (602.25pt,197.639pt);
	\draw[line width=1pt, line join=bevel, line cap=rect](595.475pt,259.369pt) -- (602.25pt,259.369pt);
	\draw[line width=1pt, line join=bevel, line cap=rect](595.475pt,320.347pt) -- (602.25pt,320.347pt);
	\draw[line width=1pt, line join=bevel, line cap=rect](595.475pt,382.078pt) -- (602.25pt,382.078pt);
	\draw[line width=1pt, line join=bevel, line cap=rect](595.475pt,443.056pt) -- (602.25pt,443.056pt);
	\draw[line width=1pt, line join=bevel, line cap=rect](595.475pt,504.786pt) -- (602.25pt,504.786pt);
	\draw[line width=1pt, line join=bevel, line cap=rect](595.475pt,504.786pt) -- (595.475pt,197.639pt);
	\pgftext[center, base, at={\pgfpoint{333.866pt}{153.976pt}}]{\fontsize{12}{0}\selectfont{z [digit]}}
	\color[rgb]{0.129412,0.129412,0.129412}
	\pgftext[center, base, at={\pgfpoint{60.5955pt}{172.796pt}}]{\fontsize{12}{0}\selectfont{0}}
	\pgftext[center, base, at={\pgfpoint{190.462pt}{172.796pt}}]{\fontsize{12}{0}\selectfont{1,000}}
	\pgftext[center, base, at={\pgfpoint{321.451pt}{172.796pt}}]{\fontsize{12}{0}\selectfont{2,000}}
	\pgftext[center, base, at={\pgfpoint{451.688pt}{172.796pt}}]{\fontsize{12}{0}\selectfont{3,000}}
	\pgftext[center, base, at={\pgfpoint{582.677pt}{172.796pt}}]{\fontsize{12}{0}\selectfont{4,000}}
	\color[rgb]{0,0,0}
	\draw[line width=1pt, line join=bevel, line cap=rect](86.5734pt,197.639pt) -- (86.5734pt,193.875pt);
	\draw[line width=1pt, line join=bevel, line cap=rect](112.169pt,197.639pt) -- (112.169pt,193.875pt);
	\draw[line width=1pt, line join=bevel, line cap=rect](138.517pt,197.639pt) -- (138.517pt,193.875pt);
	\draw[line width=1pt, line join=bevel, line cap=rect](164.866pt,197.639pt) -- (164.866pt,193.875pt);
	\draw[line width=1pt, line join=bevel, line cap=rect](216.81pt,197.639pt) -- (216.81pt,193.875pt);
	\draw[line width=1pt, line join=bevel, line cap=rect](243.158pt,197.639pt) -- (243.158pt,193.875pt);
	\draw[line width=1pt, line join=bevel, line cap=rect](268.754pt,197.639pt) -- (268.754pt,193.875pt);
	\draw[line width=1pt, line join=bevel, line cap=rect](295.102pt,197.639pt) -- (295.102pt,193.875pt);
	\draw[line width=1pt, line join=bevel, line cap=rect](347.799pt,197.639pt) -- (347.799pt,193.875pt);
	\draw[line width=1pt, line join=bevel, line cap=rect](373.395pt,197.639pt) -- (373.395pt,193.875pt);
	\draw[line width=1pt, line join=bevel, line cap=rect](399.743pt,197.639pt) -- (399.743pt,193.875pt);
	\draw[line width=1pt, line join=bevel, line cap=rect](426.092pt,197.639pt) -- (426.092pt,193.875pt);
	\draw[line width=1pt, line join=bevel, line cap=rect](478.036pt,197.639pt) -- (478.036pt,193.875pt);
	\draw[line width=1pt, line join=bevel, line cap=rect](504.384pt,197.639pt) -- (504.384pt,193.875pt);
	\draw[line width=1pt, line join=bevel, line cap=rect](529.98pt,197.639pt) -- (529.98pt,193.875pt);
	\draw[line width=1pt, line join=bevel, line cap=rect](556.328pt,197.639pt) -- (556.328pt,193.875pt);
	\draw[line width=1pt, line join=bevel, line cap=rect](60.225pt,197.639pt) -- (60.225pt,190.863pt);
	\draw[line width=1pt, line join=bevel, line cap=rect](190.462pt,197.639pt) -- (190.462pt,190.863pt);
	\draw[line width=1pt, line join=bevel, line cap=rect](321.451pt,197.639pt) -- (321.451pt,190.863pt);
	\draw[line width=1pt, line join=bevel, line cap=rect](451.688pt,197.639pt) -- (451.688pt,190.863pt);
	\draw[line width=1pt, line join=bevel, line cap=rect](582.677pt,197.639pt) -- (582.677pt,190.863pt);
	\draw[line width=1pt, line join=bevel, line cap=rect](60.225pt,197.639pt) -- (595.475pt,197.639pt);
	\color[rgb]{0.129412,0.129412,0.129412}
	\pgftext[center, base, at={\pgfpoint{60.5955pt}{519.09pt}}]{\fontsize{12}{0}\selectfont{0}}
	\pgftext[center, base, at={\pgfpoint{190.462pt}{519.09pt}}]{\fontsize{12}{0}\selectfont{1,000}}
	\pgftext[center, base, at={\pgfpoint{321.451pt}{519.09pt}}]{\fontsize{12}{0}\selectfont{2,000}}
	\pgftext[center, base, at={\pgfpoint{451.688pt}{519.09pt}}]{\fontsize{12}{0}\selectfont{3,000}}
	\pgftext[center, base, at={\pgfpoint{582.677pt}{519.09pt}}]{\fontsize{12}{0}\selectfont{4,000}}
	\color[rgb]{0,0,0}
	\draw[line width=1pt, line join=bevel, line cap=rect](86.5734pt,504.786pt) -- (86.5734pt,508.55pt);
	\draw[line width=1pt, line join=bevel, line cap=rect](112.169pt,504.786pt) -- (112.169pt,508.55pt);
	\draw[line width=1pt, line join=bevel, line cap=rect](138.517pt,504.786pt) -- (138.517pt,508.55pt);
	\draw[line width=1pt, line join=bevel, line cap=rect](164.866pt,504.786pt) -- (164.866pt,508.55pt);
	\draw[line width=1pt, line join=bevel, line cap=rect](216.81pt,504.786pt) -- (216.81pt,508.55pt);
	\draw[line width=1pt, line join=bevel, line cap=rect](243.158pt,504.786pt) -- (243.158pt,508.55pt);
	\draw[line width=1pt, line join=bevel, line cap=rect](268.754pt,504.786pt) -- (268.754pt,508.55pt);
	\draw[line width=1pt, line join=bevel, line cap=rect](295.102pt,504.786pt) -- (295.102pt,508.55pt);
	\draw[line width=1pt, line join=bevel, line cap=rect](347.799pt,504.786pt) -- (347.799pt,508.55pt);
	\draw[line width=1pt, line join=bevel, line cap=rect](373.395pt,504.786pt) -- (373.395pt,508.55pt);
	\draw[line width=1pt, line join=bevel, line cap=rect](399.743pt,504.786pt) -- (399.743pt,508.55pt);
	\draw[line width=1pt, line join=bevel, line cap=rect](426.092pt,504.786pt) -- (426.092pt,508.55pt);
	\draw[line width=1pt, line join=bevel, line cap=rect](478.036pt,504.786pt) -- (478.036pt,508.55pt);
	\draw[line width=1pt, line join=bevel, line cap=rect](504.384pt,504.786pt) -- (504.384pt,508.55pt);
	\draw[line width=1pt, line join=bevel, line cap=rect](529.98pt,504.786pt) -- (529.98pt,508.55pt);
	\draw[line width=1pt, line join=bevel, line cap=rect](556.328pt,504.786pt) -- (556.328pt,508.55pt);
	\draw[line width=1pt, line join=bevel, line cap=rect](60.225pt,504.786pt) -- (60.225pt,511.562pt);
	\draw[line width=1pt, line join=bevel, line cap=rect](190.462pt,504.786pt) -- (190.462pt,511.562pt);
	\draw[line width=1pt, line join=bevel, line cap=rect](321.451pt,504.786pt) -- (321.451pt,511.562pt);
	\draw[line width=1pt, line join=bevel, line cap=rect](451.688pt,504.786pt) -- (451.688pt,511.562pt);
	\draw[line width=1pt, line join=bevel, line cap=rect](582.677pt,504.786pt) -- (582.677pt,511.562pt);
	\draw[line width=1pt, line join=bevel, line cap=rect](60.225pt,504.786pt) -- (595.475pt,504.786pt);
\end{scope}
\end{tikzpicture}

 				}
 			\end{figure}
 		\end{frame}
 		
 		\begin{frame}{Messergebnis: integrale Steuerstabkennlinie}
 		 	\begin{figure}[hp]
 		 		\centering
 		 		\scalebox{0.4}{
 		 		\begin{tikzpicture}{0pt}{0pt}{872pt}{562pt}
	\clip(0pt,562pt) -- (656.452pt,562pt) -- (656.452pt,138.919pt) -- (0pt,138.919pt) -- (0pt,562pt);
\begin{scope}
	\clip(53.4497pt,504.786pt) -- (601.497pt,504.786pt) -- (601.497pt,196.133pt) -- (53.4497pt,196.133pt) -- (53.4497pt,504.786pt);
	\color[rgb]{0.627451,0.627451,0.643137}
	\draw[line width=0.4pt, dash pattern=on 0.0096cm off 0.032cm, dash phase=0pt, line join=bevel, line cap=rect](80.5509pt,504.786pt) -- (80.5509pt,196.886pt);
	\color[rgb]{0.627451,0.627451,0.643137}
	\draw[line width=0.4pt, dash pattern=on 0.0096cm off 0.032cm, dash phase=0pt, line join=bevel, line cap=rect](106.899pt,504.786pt) -- (106.899pt,196.886pt);
	\draw[line width=0.4pt, dash pattern=on 0.0096cm off 0.032cm, dash phase=0pt, line join=bevel, line cap=rect](134.001pt,504.786pt) -- (134.001pt,196.886pt);
	\draw[line width=0.4pt, dash pattern=on 0.0096cm off 0.032cm, dash phase=0pt, line join=bevel, line cap=rect](160.349pt,504.786pt) -- (160.349pt,196.886pt);
	\draw[line width=0.4pt, dash pattern=on 0.0096cm off 0.032cm, dash phase=0pt, line join=bevel, line cap=rect](213.799pt,504.786pt) -- (213.799pt,196.886pt);
	\draw[line width=0.4pt, dash pattern=on 0.0096cm off 0.032cm, dash phase=0pt, line join=bevel, line cap=rect](240.9pt,504.786pt) -- (240.9pt,196.886pt);
	\draw[line width=0.4pt, dash pattern=on 0.0096cm off 0.032cm, dash phase=0pt, line join=bevel, line cap=rect](267.248pt,504.786pt) -- (267.248pt,196.886pt);
	\draw[line width=0.4pt, dash pattern=on 0.0096cm off 0.032cm, dash phase=0pt, line join=bevel, line cap=rect](294.35pt,504.786pt) -- (294.35pt,196.886pt);
	\draw[line width=0.4pt, dash pattern=on 0.0096cm off 0.032cm, dash phase=0pt, line join=bevel, line cap=rect](347.799pt,504.786pt) -- (347.799pt,196.886pt);
	\draw[line width=0.4pt, dash pattern=on 0.0096cm off 0.032cm, dash phase=0pt, line join=bevel, line cap=rect](374.148pt,504.786pt) -- (374.148pt,196.886pt);
	\draw[line width=0.4pt, dash pattern=on 0.0096cm off 0.032cm, dash phase=0pt, line join=bevel, line cap=rect](401.249pt,504.786pt) -- (401.249pt,196.886pt);
	\draw[line width=0.4pt, dash pattern=on 0.0096cm off 0.032cm, dash phase=0pt, line join=bevel, line cap=rect](427.597pt,504.786pt) -- (427.597pt,196.886pt);
	\draw[line width=0.4pt, dash pattern=on 0.0096cm off 0.032cm, dash phase=0pt, line join=bevel, line cap=rect](481.047pt,504.786pt) -- (481.047pt,196.886pt);
	\draw[line width=0.4pt, dash pattern=on 0.0096cm off 0.032cm, dash phase=0pt, line join=bevel, line cap=rect](508.148pt,504.786pt) -- (508.148pt,196.886pt);
	\draw[line width=0.4pt, dash pattern=on 0.0096cm off 0.032cm, dash phase=0pt, line join=bevel, line cap=rect](534.497pt,504.786pt) -- (534.497pt,196.886pt);
	\draw[line width=0.4pt, dash pattern=on 0.0096cm off 0.032cm, dash phase=0pt, line join=bevel, line cap=rect](561.598pt,504.786pt) -- (561.598pt,196.886pt);
	\draw[line width=0.4pt, dash pattern=on 0.0096cm off 0.032cm, dash phase=0pt, line join=bevel, line cap=rect](53.4497pt,211.189pt) -- (600.744pt,211.189pt);
	\draw[line width=0.4pt, dash pattern=on 0.0096cm off 0.032cm, dash phase=0pt, line join=bevel, line cap=rect](53.4497pt,226.998pt) -- (600.744pt,226.998pt);
	\draw[line width=0.4pt, dash pattern=on 0.0096cm off 0.032cm, dash phase=0pt, line join=bevel, line cap=rect](53.4497pt,242.055pt) -- (600.744pt,242.055pt);
	\draw[line width=0.4pt, dash pattern=on 0.0096cm off 0.032cm, dash phase=0pt, line join=bevel, line cap=rect](53.4497pt,257.864pt) -- (600.744pt,257.864pt);
	\draw[line width=0.4pt, dash pattern=on 0.0096cm off 0.032cm, dash phase=0pt, line join=bevel, line cap=rect](53.4497pt,288.729pt) -- (600.744pt,288.729pt);
	\draw[line width=0.4pt, dash pattern=on 0.0096cm off 0.032cm, dash phase=0pt, line join=bevel, line cap=rect](53.4497pt,303.785pt) -- (600.744pt,303.785pt);
	\draw[line width=0.4pt, dash pattern=on 0.0096cm off 0.032cm, dash phase=0pt, line join=bevel, line cap=rect](53.4497pt,319.594pt) -- (600.744pt,319.594pt);
	\draw[line width=0.4pt, dash pattern=on 0.0096cm off 0.032cm, dash phase=0pt, line join=bevel, line cap=rect](53.4497pt,334.651pt) -- (600.744pt,334.651pt);
	\draw[line width=0.4pt, dash pattern=on 0.0096cm off 0.032cm, dash phase=0pt, line join=bevel, line cap=rect](53.4497pt,365.516pt) -- (600.744pt,365.516pt);
	\draw[line width=0.4pt, dash pattern=on 0.0096cm off 0.032cm, dash phase=0pt, line join=bevel, line cap=rect](53.4497pt,381.325pt) -- (600.744pt,381.325pt);
	\draw[line width=0.4pt, dash pattern=on 0.0096cm off 0.032cm, dash phase=0pt, line join=bevel, line cap=rect](53.4497pt,396.381pt) -- (600.744pt,396.381pt);
	\draw[line width=0.4pt, dash pattern=on 0.0096cm off 0.032cm, dash phase=0pt, line join=bevel, line cap=rect](53.4497pt,412.19pt) -- (600.744pt,412.19pt);
	\draw[line width=0.4pt, dash pattern=on 0.0096cm off 0.032cm, dash phase=0pt, line join=bevel, line cap=rect](53.4497pt,443.056pt) -- (600.744pt,443.056pt);
	\draw[line width=0.4pt, dash pattern=on 0.0096cm off 0.032cm, dash phase=0pt, line join=bevel, line cap=rect](53.4497pt,458.865pt) -- (600.744pt,458.865pt);
	\draw[line width=0.4pt, dash pattern=on 0.0096cm off 0.032cm, dash phase=0pt, line join=bevel, line cap=rect](53.4497pt,473.921pt) -- (600.744pt,473.921pt);
	\draw[line width=0.4pt, dash pattern=on 0.0096cm off 0.032cm, dash phase=0pt, line join=bevel, line cap=rect](53.4497pt,489.73pt) -- (600.744pt,489.73pt);
	\color[rgb]{0,0,1}
	\draw[line width=0.5pt, line join=bevel, line cap=rect](187.45pt,504.786pt) -- (187.45pt,196.886pt);
	\draw[line width=0.5pt, line join=bevel, line cap=rect](320.698pt,504.786pt) -- (320.698pt,196.886pt);
	\draw[line width=0.5pt, line join=bevel, line cap=rect](454.699pt,504.786pt) -- (454.699pt,196.886pt);
	\draw[line width=0.5pt, line join=bevel, line cap=rect](587.947pt,504.786pt) -- (587.947pt,196.886pt);
	\draw[line width=0.5pt, line join=bevel, line cap=rect](53.4497pt,272.92pt) -- (600.744pt,272.92pt);
	\draw[line width=0.5pt, line join=bevel, line cap=rect](53.4497pt,350.46pt) -- (600.744pt,350.46pt);
	\draw[line width=0.5pt, line join=bevel, line cap=rect](53.4497pt,427.999pt) -- (600.744pt,427.999pt);
	\color[rgb]{0,0,0}
	\draw[line width=1pt, dash pattern=on 0.12cm off 0.08cm, dash phase=0pt, line join=miter, line cap=rect](53.4497pt,196.133pt) -- (162.458pt,243.55pt) -- (289.712pt,302.04pt) -- (388.628pt,376.116pt) -- (495.63pt,434.606pt) -- (588.13pt,482.023pt);
	\draw[line width=1pt, line join=miter, line cap=rect](162.607pt,245.819pt) -- (162.607pt,241.302pt);
	\draw[line width=1pt, line join=miter, line cap=rect](160.349pt,243.56pt) -- (164.866pt,243.56pt);
	\draw[line width=1pt, line join=miter, line cap=rect](289.833pt,304.538pt) -- (289.833pt,300.021pt);
	\draw[line width=1pt, line join=miter, line cap=rect](287.574pt,302.28pt) -- (292.091pt,302.28pt);
	\draw[line width=1pt, line join=miter, line cap=rect](388.451pt,378.314pt) -- (388.451pt,373.797pt);
	\draw[line width=1pt, line join=miter, line cap=rect](386.193pt,376.055pt) -- (390.71pt,376.055pt);
	\draw[line width=1pt, line join=miter, line cap=rect](495.351pt,437.033pt) -- (495.351pt,432.516pt);
	\draw[line width=1pt, line join=miter, line cap=rect](493.092pt,434.775pt) -- (497.609pt,434.775pt);
	\draw[line width=1pt, line join=miter, line cap=rect](587.947pt,484.46pt) -- (587.947pt,479.943pt);
	\draw[line width=1pt, line join=miter, line cap=rect](585.688pt,482.202pt) -- (590.205pt,482.202pt);
	\color[rgb]{1,0,0}
	\draw[line width=1pt, dash pattern=on 0.12cm off 0.08cmon 0.024cm off 0.08cm, dash phase=0pt, line join=miter, line cap=rect](53.4497pt,196.133pt) -- (160.92pt,244.515pt) -- (310.898pt,303.236pt) -- (404.467pt,377.313pt) -- (504.052pt,435.571pt) -- (588.13pt,482.023pt);
	\draw[line width=1pt, line join=miter, line cap=rect](161.102pt,246.572pt) -- (161.102pt,242.055pt);
	\draw[line width=1pt, line join=miter, line cap=rect](158.843pt,244.313pt) -- (163.36pt,244.313pt);
	\draw[line width=1pt, line join=miter, line cap=rect](310.912pt,305.291pt) -- (310.912pt,300.774pt);
	\draw[line width=1pt, line join=miter, line cap=rect](308.653pt,303.033pt) -- (313.17pt,303.033pt);
	\draw[line width=1pt, line join=miter, line cap=rect](404.26pt,379.819pt) -- (404.26pt,375.303pt);
	\draw[line width=1pt, line join=miter, line cap=rect](402.002pt,377.561pt) -- (406.519pt,377.561pt);
	\draw[line width=1pt, line join=miter, line cap=rect](504.384pt,437.786pt) -- (504.384pt,433.269pt);
	\draw[line width=1pt, line join=miter, line cap=rect](502.126pt,435.528pt) -- (506.643pt,435.528pt);
	\draw[line width=1pt, line join=miter, line cap=rect](587.947pt,484.46pt) -- (587.947pt,479.943pt);
	\draw[line width=1pt, line join=miter, line cap=rect](585.688pt,482.202pt) -- (590.205pt,482.202pt);
	\color[rgb]{0,1,0}
	\draw[line width=1pt, line join=miter, line cap=rect](53.4497pt,196.133pt) -- (163.995pt,242.585pt) -- (268.525pt,300.844pt) -- (372.788pt,374.92pt) -- (487.209pt,433.642pt) -- (588.13pt,482.023pt);
	\draw[line width=1pt, line join=miter, line cap=rect](164.113pt,245.066pt) -- (164.113pt,240.549pt);
	\draw[line width=1pt, line join=miter, line cap=rect](161.855pt,242.808pt) -- (166.372pt,242.808pt);
	\draw[line width=1pt, line join=miter, line cap=rect](268.754pt,303.033pt) -- (268.754pt,298.516pt);
	\draw[line width=1pt, line join=miter, line cap=rect](266.496pt,300.774pt) -- (271.012pt,300.774pt);
	\draw[line width=1pt, line join=miter, line cap=rect](372.642pt,376.808pt) -- (372.642pt,372.291pt);
	\draw[line width=1pt, line join=miter, line cap=rect](370.384pt,374.55pt) -- (374.901pt,374.55pt);
	\draw[line width=1pt, line join=miter, line cap=rect](487.07pt,435.528pt) -- (487.07pt,431.011pt);
	\draw[line width=1pt, line join=miter, line cap=rect](484.811pt,433.269pt) -- (489.328pt,433.269pt);
	\draw[line width=1pt, line join=miter, line cap=rect](587.947pt,484.46pt) -- (587.947pt,479.943pt);
	\draw[line width=1pt, line join=miter, line cap=rect](585.688pt,482.202pt) -- (590.205pt,482.202pt);
	\draw[line width=1pt, line join=miter, line cap=rect](163.995pt,244.844pt) -- (163.995pt,243.357pt);
	\draw[line width=1pt, line join=miter, line cap=rect](160.984pt,243.357pt) -- (167.006pt,243.357pt);
	\draw[line width=1pt, line join=miter, line cap=rect](163.995pt,240.327pt) -- (163.995pt,241.814pt);
	\draw[line width=1pt, line join=miter, line cap=rect](160.984pt,241.814pt) -- (167.006pt,241.814pt);
	\draw[line width=1pt, line join=miter, line cap=rect](163.995pt,244.844pt) -- (163.995pt,240.327pt);
	\draw[line width=1pt, line join=miter, line cap=rect](268.525pt,303.102pt) -- (268.525pt,301.615pt);
	\draw[line width=1pt, line join=miter, line cap=rect](265.514pt,301.615pt) -- (271.536pt,301.615pt);
	\draw[line width=1pt, line join=miter, line cap=rect](268.525pt,298.585pt) -- (268.525pt,300.072pt);
	\draw[line width=1pt, line join=miter, line cap=rect](265.514pt,300.072pt) -- (271.536pt,300.072pt);
	\draw[line width=1pt, line join=miter, line cap=rect](268.525pt,303.102pt) -- (268.525pt,298.585pt);
	\draw[line width=1pt, line join=miter, line cap=rect](372.788pt,377.179pt) -- (372.788pt,376.464pt);
	\draw[line width=1pt, line join=miter, line cap=rect](369.776pt,376.464pt) -- (375.799pt,376.464pt);
	\draw[line width=1pt, line join=miter, line cap=rect](372.788pt,372.662pt) -- (372.788pt,373.377pt);
	\draw[line width=1pt, line join=miter, line cap=rect](369.776pt,373.377pt) -- (375.799pt,373.377pt);
	\draw[line width=1pt, line join=miter, line cap=rect](372.788pt,377.179pt) -- (372.788pt,372.662pt);
	\draw[line width=1pt, line join=miter, line cap=rect](487.209pt,435.9pt) -- (487.209pt,435.957pt);
	\draw[line width=1pt, line join=miter, line cap=rect](484.198pt,435.957pt) -- (490.22pt,435.957pt);
	\draw[line width=1pt, line join=miter, line cap=rect](487.209pt,431.383pt) -- (487.209pt,431.327pt);
	\draw[line width=1pt, line join=miter, line cap=rect](484.198pt,431.327pt) -- (490.22pt,431.327pt);
	\draw[line width=1pt, line join=miter, line cap=rect](487.209pt,435.9pt) -- (487.209pt,431.383pt);
	\draw[line width=1pt, line join=miter, line cap=rect](588.13pt,484.282pt) -- (588.13pt,484.338pt);
	\draw[line width=1pt, line join=miter, line cap=rect](585.119pt,484.338pt) -- (591.141pt,484.338pt);
	\draw[line width=1pt, line join=miter, line cap=rect](588.13pt,479.765pt) -- (588.13pt,479.708pt);
	\draw[line width=1pt, line join=miter, line cap=rect](585.119pt,479.708pt) -- (591.141pt,479.708pt);
	\draw[line width=1pt, line join=miter, line cap=rect](588.13pt,484.282pt) -- (588.13pt,479.765pt);
	\color[rgb]{1,0,0}
	\draw[line width=1pt, line join=miter, line cap=rect](160.92pt,246.773pt) -- (160.92pt,245.286pt);
	\draw[line width=1pt, line join=miter, line cap=rect](157.909pt,245.286pt) -- (163.932pt,245.286pt);
	\draw[line width=1pt, line join=miter, line cap=rect](160.92pt,242.256pt) -- (160.92pt,243.743pt);
	\draw[line width=1pt, line join=miter, line cap=rect](157.909pt,243.743pt) -- (163.932pt,243.743pt);
	\draw[line width=1pt, line join=miter, line cap=rect](160.92pt,246.773pt) -- (160.92pt,242.256pt);
	\draw[line width=1pt, line join=miter, line cap=rect](310.898pt,305.494pt) -- (310.898pt,304.007pt);
	\draw[line width=1pt, line join=miter, line cap=rect](307.887pt,304.007pt) -- (313.91pt,304.007pt);
	\draw[line width=1pt, line join=miter, line cap=rect](310.898pt,300.977pt) -- (310.898pt,302.464pt);
	\draw[line width=1pt, line join=miter, line cap=rect](307.887pt,302.464pt) -- (313.91pt,302.464pt);
	\draw[line width=1pt, line join=miter, line cap=rect](310.898pt,305.494pt) -- (310.898pt,300.977pt);
	\draw[line width=1pt, line join=miter, line cap=rect](404.467pt,379.571pt) -- (404.467pt,378.856pt);
	\draw[line width=1pt, line join=miter, line cap=rect](401.456pt,378.856pt) -- (407.479pt,378.856pt);
	\draw[line width=1pt, line join=miter, line cap=rect](404.467pt,375.054pt) -- (404.467pt,375.769pt);
	\draw[line width=1pt, line join=miter, line cap=rect](401.456pt,375.769pt) -- (407.479pt,375.769pt);
	\draw[line width=1pt, line join=miter, line cap=rect](404.467pt,379.571pt) -- (404.467pt,375.054pt);
	\draw[line width=1pt, line join=miter, line cap=rect](504.052pt,437.829pt) -- (504.052pt,437.886pt);
	\draw[line width=1pt, line join=miter, line cap=rect](501.04pt,437.886pt) -- (507.063pt,437.886pt);
	\draw[line width=1pt, line join=miter, line cap=rect](504.052pt,433.312pt) -- (504.052pt,433.256pt);
	\draw[line width=1pt, line join=miter, line cap=rect](501.04pt,433.256pt) -- (507.063pt,433.256pt);
	\draw[line width=1pt, line join=miter, line cap=rect](504.052pt,437.829pt) -- (504.052pt,433.312pt);
	\draw[line width=1pt, line join=miter, line cap=rect](588.13pt,484.282pt) -- (588.13pt,484.338pt);
	\draw[line width=1pt, line join=miter, line cap=rect](585.119pt,484.338pt) -- (591.141pt,484.338pt);
	\draw[line width=1pt, line join=miter, line cap=rect](588.13pt,479.765pt) -- (588.13pt,479.708pt);
	\draw[line width=1pt, line join=miter, line cap=rect](585.119pt,479.708pt) -- (591.141pt,479.708pt);
	\draw[line width=1pt, line join=miter, line cap=rect](588.13pt,484.282pt) -- (588.13pt,479.765pt);
	\color[rgb]{0,0,0}
	\draw[line width=1pt, line join=miter, line cap=rect](162.458pt,245.808pt) -- (162.458pt,255.757pt);
	\draw[line width=1pt, line join=miter, line cap=rect](159.446pt,255.757pt) -- (165.469pt,255.757pt);
	\draw[line width=1pt, line join=miter, line cap=rect](162.458pt,241.292pt) -- (162.458pt,231.343pt);
	\draw[line width=1pt, line join=miter, line cap=rect](159.446pt,231.343pt) -- (165.469pt,231.343pt);
	\draw[line width=1pt, line join=miter, line cap=rect](162.458pt,245.808pt) -- (162.458pt,241.292pt);
	\draw[line width=1pt, line join=miter, line cap=rect](289.712pt,304.298pt) -- (289.712pt,314.246pt);
	\draw[line width=1pt, line join=miter, line cap=rect](286.7pt,314.246pt) -- (292.723pt,314.246pt);
	\draw[line width=1pt, line join=miter, line cap=rect](289.712pt,299.781pt) -- (289.712pt,289.833pt);
	\draw[line width=1pt, line join=miter, line cap=rect](286.7pt,289.833pt) -- (292.723pt,289.833pt);
	\draw[line width=1pt, line join=miter, line cap=rect](289.712pt,304.298pt) -- (289.712pt,299.781pt);
	\draw[line width=1pt, line join=miter, line cap=rect](388.628pt,378.375pt) -- (388.628pt,393.388pt);
	\draw[line width=1pt, line join=miter, line cap=rect](385.616pt,393.388pt) -- (391.639pt,393.388pt);
	\draw[line width=1pt, line join=miter, line cap=rect](388.628pt,373.858pt) -- (388.628pt,358.845pt);
	\draw[line width=1pt, line join=miter, line cap=rect](385.616pt,358.845pt) -- (391.639pt,358.845pt);
	\draw[line width=1pt, line join=miter, line cap=rect](388.628pt,378.375pt) -- (388.628pt,373.858pt);
	\draw[line width=1pt, line join=miter, line cap=rect](495.63pt,436.865pt) -- (495.63pt,455.77pt);
	\draw[line width=1pt, line join=miter, line cap=rect](492.619pt,455.77pt) -- (498.642pt,455.77pt);
	\draw[line width=1pt, line join=miter, line cap=rect](495.63pt,432.348pt) -- (495.63pt,413.443pt);
	\draw[line width=1pt, line join=miter, line cap=rect](492.619pt,413.443pt) -- (498.642pt,413.443pt);
	\draw[line width=1pt, line join=miter, line cap=rect](495.63pt,436.865pt) -- (495.63pt,432.348pt);
	\draw[line width=1pt, line join=miter, line cap=rect](588.13pt,484.282pt) -- (588.13pt,503.187pt);
	\draw[line width=1pt, line join=miter, line cap=rect](585.119pt,503.187pt) -- (591.141pt,503.187pt);
	\draw[line width=1pt, line join=miter, line cap=rect](588.13pt,479.765pt) -- (588.13pt,460.859pt);
	\draw[line width=1pt, line join=miter, line cap=rect](585.119pt,460.859pt) -- (591.141pt,460.859pt);
	\draw[line width=1pt, line join=miter, line cap=rect](588.13pt,484.282pt) -- (588.13pt,479.765pt);
\end{scope}
\begin{scope}
	\color[rgb]{0,0,0}
	\pgftext[center, base, at={\pgfpoint{327.844pt}{540.921pt}}]{\fontsize{14}{0}\selectfont{\textbf{Integrale Steuerstabkennlinie}}}
	\color[rgb]{0,0,0}
	\pgftext[center, base, at={\pgfpoint{19.5731pt}{339.35pt}},rotate=90]{\fontsize{12}{0}\selectfont{$\rho\ [\unit{\$}] $}}
	\color[rgb]{0.129412,0.129412,0.129412}
	\pgftext[center, base, at={\pgfpoint{38.8287pt}{191.616pt}}]{\fontsize{12}{0}\selectfont{0}}
	\pgftext[center, base, at={\pgfpoint{34.8529pt}{268.403pt}}]{\fontsize{12}{0}\selectfont{0.1}}
	\pgftext[center, base, at={\pgfpoint{34.1001pt}{345.943pt}}]{\fontsize{12}{0}\selectfont{0.2}}
	\pgftext[center, base, at={\pgfpoint{34.1001pt}{423.483pt}}]{\fontsize{12}{0}\selectfont{0.3}}
	\pgftext[center, base, at={\pgfpoint{34.1001pt}{500.269pt}}]{\fontsize{12}{0}\selectfont{0.4}}
	\color[rgb]{0,0,0}
	\draw[line width=1pt, line join=bevel, line cap=rect](53.4497pt,211.189pt) -- (49.6856pt,211.189pt);
	\draw[line width=1pt, line join=bevel, line cap=rect](53.4497pt,226.998pt) -- (49.6856pt,226.998pt);
	\draw[line width=1pt, line join=bevel, line cap=rect](53.4497pt,242.055pt) -- (49.6856pt,242.055pt);
	\draw[line width=1pt, line join=bevel, line cap=rect](53.4497pt,257.864pt) -- (49.6856pt,257.864pt);
	\draw[line width=1pt, line join=bevel, line cap=rect](53.4497pt,288.729pt) -- (49.6856pt,288.729pt);
	\draw[line width=1pt, line join=bevel, line cap=rect](53.4497pt,303.785pt) -- (49.6856pt,303.785pt);
	\draw[line width=1pt, line join=bevel, line cap=rect](53.4497pt,319.594pt) -- (49.6856pt,319.594pt);
	\draw[line width=1pt, line join=bevel, line cap=rect](53.4497pt,334.651pt) -- (49.6856pt,334.651pt);
	\draw[line width=1pt, line join=bevel, line cap=rect](53.4497pt,365.516pt) -- (49.6856pt,365.516pt);
	\draw[line width=1pt, line join=bevel, line cap=rect](53.4497pt,381.325pt) -- (49.6856pt,381.325pt);
	\draw[line width=1pt, line join=bevel, line cap=rect](53.4497pt,396.381pt) -- (49.6856pt,396.381pt);
	\draw[line width=1pt, line join=bevel, line cap=rect](53.4497pt,412.19pt) -- (49.6856pt,412.19pt);
	\draw[line width=1pt, line join=bevel, line cap=rect](53.4497pt,443.056pt) -- (49.6856pt,443.056pt);
	\draw[line width=1pt, line join=bevel, line cap=rect](53.4497pt,458.865pt) -- (49.6856pt,458.865pt);
	\draw[line width=1pt, line join=bevel, line cap=rect](53.4497pt,473.921pt) -- (49.6856pt,473.921pt);
	\draw[line width=1pt, line join=bevel, line cap=rect](53.4497pt,489.73pt) -- (49.6856pt,489.73pt);
	\draw[line width=1pt, line join=bevel, line cap=rect](53.4497pt,196.133pt) -- (46.6744pt,196.133pt);
	\draw[line width=1pt, line join=bevel, line cap=rect](53.4497pt,272.92pt) -- (46.6744pt,272.92pt);
	\draw[line width=1pt, line join=bevel, line cap=rect](53.4497pt,350.46pt) -- (46.6744pt,350.46pt);
	\draw[line width=1pt, line join=bevel, line cap=rect](53.4497pt,427.999pt) -- (46.6744pt,427.999pt);
	\draw[line width=1pt, line join=bevel, line cap=rect](53.4497pt,504.786pt) -- (46.6744pt,504.786pt);
	\draw[line width=1pt, line join=bevel, line cap=rect](53.4497pt,504.786pt) -- (53.4497pt,196.133pt);
	\color[rgb]{0.129412,0.129412,0.129412}
	\pgftext[center, base, at={\pgfpoint{615.483pt}{191.616pt}}]{\fontsize{12}{0}\selectfont{0}}
	\pgftext[center, base, at={\pgfpoint{620.541pt}{268.403pt}}]{\fontsize{12}{0}\selectfont{0.1}}
	\pgftext[center, base, at={\pgfpoint{621.294pt}{345.943pt}}]{\fontsize{12}{0}\selectfont{0.2}}
	\pgftext[center, base, at={\pgfpoint{621.294pt}{423.483pt}}]{\fontsize{12}{0}\selectfont{0.3}}
	\pgftext[center, base, at={\pgfpoint{621.294pt}{500.269pt}}]{\fontsize{12}{0}\selectfont{0.4}}
	\color[rgb]{0,0,0}
	\draw[line width=1pt, line join=bevel, line cap=rect](601.497pt,211.189pt) -- (605.261pt,211.189pt);
	\draw[line width=1pt, line join=bevel, line cap=rect](601.497pt,226.998pt) -- (605.261pt,226.998pt);
	\draw[line width=1pt, line join=bevel, line cap=rect](601.497pt,242.055pt) -- (605.261pt,242.055pt);
	\draw[line width=1pt, line join=bevel, line cap=rect](601.497pt,257.864pt) -- (605.261pt,257.864pt);
	\draw[line width=1pt, line join=bevel, line cap=rect](601.497pt,288.729pt) -- (605.261pt,288.729pt);
	\draw[line width=1pt, line join=bevel, line cap=rect](601.497pt,303.785pt) -- (605.261pt,303.785pt);
	\draw[line width=1pt, line join=bevel, line cap=rect](601.497pt,319.594pt) -- (605.261pt,319.594pt);
	\draw[line width=1pt, line join=bevel, line cap=rect](601.497pt,334.651pt) -- (605.261pt,334.651pt);
	\draw[line width=1pt, line join=bevel, line cap=rect](601.497pt,365.516pt) -- (605.261pt,365.516pt);
	\draw[line width=1pt, line join=bevel, line cap=rect](601.497pt,381.325pt) -- (605.261pt,381.325pt);
	\draw[line width=1pt, line join=bevel, line cap=rect](601.497pt,396.381pt) -- (605.261pt,396.381pt);
	\draw[line width=1pt, line join=bevel, line cap=rect](601.497pt,412.19pt) -- (605.261pt,412.19pt);
	\draw[line width=1pt, line join=bevel, line cap=rect](601.497pt,443.056pt) -- (605.261pt,443.056pt);
	\draw[line width=1pt, line join=bevel, line cap=rect](601.497pt,458.865pt) -- (605.261pt,458.865pt);
	\draw[line width=1pt, line join=bevel, line cap=rect](601.497pt,473.921pt) -- (605.261pt,473.921pt);
	\draw[line width=1pt, line join=bevel, line cap=rect](601.497pt,489.73pt) -- (605.261pt,489.73pt);
	\draw[line width=1pt, line join=bevel, line cap=rect](601.497pt,196.133pt) -- (608.272pt,196.133pt);
	\draw[line width=1pt, line join=bevel, line cap=rect](601.497pt,272.92pt) -- (608.272pt,272.92pt);
	\draw[line width=1pt, line join=bevel, line cap=rect](601.497pt,350.46pt) -- (608.272pt,350.46pt);
	\draw[line width=1pt, line join=bevel, line cap=rect](601.497pt,427.999pt) -- (608.272pt,427.999pt);
	\draw[line width=1pt, line join=bevel, line cap=rect](601.497pt,504.786pt) -- (608.272pt,504.786pt);
	\draw[line width=1pt, line join=bevel, line cap=rect](601.497pt,504.786pt) -- (601.497pt,196.133pt);
	\pgftext[center, base, at={\pgfpoint{333.49pt}{152.47pt}}]{\fontsize{12}{0}\selectfont{$z\ [\unit{digit}]$}}
	\color[rgb]{0.129412,0.129412,0.129412}
	\pgftext[center, base, at={\pgfpoint{53.8261pt}{171.29pt}}]{\fontsize{12}{0}\selectfont{0}}
	\pgftext[center, base, at={\pgfpoint{187.45pt}{171.29pt}}]{\fontsize{12}{0}\selectfont{1,000}}
	\pgftext[center, base, at={\pgfpoint{320.698pt}{171.29pt}}]{\fontsize{12}{0}\selectfont{2,000}}
	\pgftext[center, base, at={\pgfpoint{454.699pt}{171.29pt}}]{\fontsize{12}{0}\selectfont{3,000}}
	\pgftext[center, base, at={\pgfpoint{587.947pt}{171.29pt}}]{\fontsize{12}{0}\selectfont{4,000}}
	\color[rgb]{0,0,0}
	\draw[line width=1pt, line join=bevel, line cap=rect](80.5509pt,196.133pt) -- (80.5509pt,192.369pt);
	\draw[line width=1pt, line join=bevel, line cap=rect](106.899pt,196.133pt) -- (106.899pt,192.369pt);
	\draw[line width=1pt, line join=bevel, line cap=rect](134.001pt,196.133pt) -- (134.001pt,192.369pt);
	\draw[line width=1pt, line join=bevel, line cap=rect](160.349pt,196.133pt) -- (160.349pt,192.369pt);
	\draw[line width=1pt, line join=bevel, line cap=rect](213.799pt,196.133pt) -- (213.799pt,192.369pt);
	\draw[line width=1pt, line join=bevel, line cap=rect](240.9pt,196.133pt) -- (240.9pt,192.369pt);
	\draw[line width=1pt, line join=bevel, line cap=rect](267.248pt,196.133pt) -- (267.248pt,192.369pt);
	\draw[line width=1pt, line join=bevel, line cap=rect](294.35pt,196.133pt) -- (294.35pt,192.369pt);
	\draw[line width=1pt, line join=bevel, line cap=rect](347.799pt,196.133pt) -- (347.799pt,192.369pt);
	\draw[line width=1pt, line join=bevel, line cap=rect](374.148pt,196.133pt) -- (374.148pt,192.369pt);
	\draw[line width=1pt, line join=bevel, line cap=rect](401.249pt,196.133pt) -- (401.249pt,192.369pt);
	\draw[line width=1pt, line join=bevel, line cap=rect](427.597pt,196.133pt) -- (427.597pt,192.369pt);
	\draw[line width=1pt, line join=bevel, line cap=rect](481.047pt,196.133pt) -- (481.047pt,192.369pt);
	\draw[line width=1pt, line join=bevel, line cap=rect](508.148pt,196.133pt) -- (508.148pt,192.369pt);
	\draw[line width=1pt, line join=bevel, line cap=rect](534.497pt,196.133pt) -- (534.497pt,192.369pt);
	\draw[line width=1pt, line join=bevel, line cap=rect](561.598pt,196.133pt) -- (561.598pt,192.369pt);
	\draw[line width=1pt, line join=bevel, line cap=rect](53.4497pt,196.133pt) -- (53.4497pt,189.358pt);
	\draw[line width=1pt, line join=bevel, line cap=rect](187.45pt,196.133pt) -- (187.45pt,189.358pt);
	\draw[line width=1pt, line join=bevel, line cap=rect](320.698pt,196.133pt) -- (320.698pt,189.358pt);
	\draw[line width=1pt, line join=bevel, line cap=rect](454.699pt,196.133pt) -- (454.699pt,189.358pt);
	\draw[line width=1pt, line join=bevel, line cap=rect](587.947pt,196.133pt) -- (587.947pt,189.358pt);
	\draw[line width=1pt, line join=bevel, line cap=rect](53.4497pt,196.133pt) -- (601.497pt,196.133pt);
	\color[rgb]{0.129412,0.129412,0.129412}
	\pgftext[center, base, at={\pgfpoint{53.8261pt}{519.09pt}}]{\fontsize{12}{0}\selectfont{0}}
	\pgftext[center, base, at={\pgfpoint{187.45pt}{519.09pt}}]{\fontsize{12}{0}\selectfont{1,000}}
	\pgftext[center, base, at={\pgfpoint{320.698pt}{519.09pt}}]{\fontsize{12}{0}\selectfont{2,000}}
	\pgftext[center, base, at={\pgfpoint{454.699pt}{519.09pt}}]{\fontsize{12}{0}\selectfont{3,000}}
	\pgftext[center, base, at={\pgfpoint{587.947pt}{519.09pt}}]{\fontsize{12}{0}\selectfont{4,000}}
	\color[rgb]{0,0,0}
	\draw[line width=1pt, line join=bevel, line cap=rect](80.5509pt,504.786pt) -- (80.5509pt,508.55pt);
	\draw[line width=1pt, line join=bevel, line cap=rect](106.899pt,504.786pt) -- (106.899pt,508.55pt);
	\draw[line width=1pt, line join=bevel, line cap=rect](134.001pt,504.786pt) -- (134.001pt,508.55pt);
	\draw[line width=1pt, line join=bevel, line cap=rect](160.349pt,504.786pt) -- (160.349pt,508.55pt);
	\draw[line width=1pt, line join=bevel, line cap=rect](213.799pt,504.786pt) -- (213.799pt,508.55pt);
	\draw[line width=1pt, line join=bevel, line cap=rect](240.9pt,504.786pt) -- (240.9pt,508.55pt);
	\draw[line width=1pt, line join=bevel, line cap=rect](267.248pt,504.786pt) -- (267.248pt,508.55pt);
	\draw[line width=1pt, line join=bevel, line cap=rect](294.35pt,504.786pt) -- (294.35pt,508.55pt);
	\draw[line width=1pt, line join=bevel, line cap=rect](347.799pt,504.786pt) -- (347.799pt,508.55pt);
	\draw[line width=1pt, line join=bevel, line cap=rect](374.148pt,504.786pt) -- (374.148pt,508.55pt);
	\draw[line width=1pt, line join=bevel, line cap=rect](401.249pt,504.786pt) -- (401.249pt,508.55pt);
	\draw[line width=1pt, line join=bevel, line cap=rect](427.597pt,504.786pt) -- (427.597pt,508.55pt);
	\draw[line width=1pt, line join=bevel, line cap=rect](481.047pt,504.786pt) -- (481.047pt,508.55pt);
	\draw[line width=1pt, line join=bevel, line cap=rect](508.148pt,504.786pt) -- (508.148pt,508.55pt);
	\draw[line width=1pt, line join=bevel, line cap=rect](534.497pt,504.786pt) -- (534.497pt,508.55pt);
	\draw[line width=1pt, line join=bevel, line cap=rect](561.598pt,504.786pt) -- (561.598pt,508.55pt);
	\draw[line width=1pt, line join=bevel, line cap=rect](53.4497pt,504.786pt) -- (53.4497pt,511.562pt);
	\draw[line width=1pt, line join=bevel, line cap=rect](187.45pt,504.786pt) -- (187.45pt,511.562pt);
	\draw[line width=1pt, line join=bevel, line cap=rect](320.698pt,504.786pt) -- (320.698pt,511.562pt);
	\draw[line width=1pt, line join=bevel, line cap=rect](454.699pt,504.786pt) -- (454.699pt,511.562pt);
	\draw[line width=1pt, line join=bevel, line cap=rect](587.947pt,504.786pt) -- (587.947pt,511.562pt);
	\draw[line width=1pt, line join=bevel, line cap=rect](53.4497pt,504.786pt) -- (601.497pt,504.786pt);
\end{scope}
\end{tikzpicture}

 		 		}
 		 	\end{figure}
 		\end{frame}
 		
 		\begin{frame}{Charakteristische Reaktivitätswerte für den AKR-2}
 			\begin{itemize}
 				\item maximale Reaktivität aller 3 Stäbe:
 				 $$\rho_{max} = \rho_{1,max} + \rho_{2,max} + \rho_{3,max} = (1,12 \pm 0,03)\ \unit{\$}$$
 				\item Überschussreaktität aller Stäbe:
 				$$\rho_{\ddot{U}berschuss} = \rho_{1,\ddot{U}berschuss} + \rho_{2,\ddot{U}berschuss} + \rho_{3,\ddot{U}berschuss}  = (0,64 \pm 0,03)\ \unit{\$}$$
 				\item Abschaltreaktivität:
 				$$\rho_{Abschalt} = \rho_{max} - \rho_{\ddot{U}berschuss} = (0,48 \pm 0,04)\ \unit{\$}$$
 			\end{itemize}
 		\end{frame}
 		
 		\begin{frame}{Bestimmung der Reaktivität einer Cadmium-Probe}
 			\begin{itemize}
 				\item ausgehend vom kritischen Zustand $(z_1,z_2,z_3) = (2.406, 2.251, 2.513)\ \unit{digit}$ wird Probe in Experimentierkanal eingebracht
 				\item unterkritischer Zustand wird durch Ausfahren von Stab 1 kompensiert
 				\item neuer kritischer Zustand $(z_1\prime,z_2\prime,z_3\prime) = (3.888, 2.251, 2.513)\ \unit{digit}$
 				\item Reaktivität  der Probe ergibt sich aus integraler Steuerstabkennlinie:
 					$$\rho_{Cd}\prime = \rho_1\prime(z_1\prime) - \rho_1\prime(z_1) = (0,13 \pm 0,02)\ \unit{\$}$$
 			\end{itemize}
 		\end{frame}
 \section{Disussion}
 	\begin{frame}{Diskussion}
 		\begin{itemize}
	 		\item die vorliegenden (biologischen) Strahlenschutzmaßnahmen genügen den geforderten Grenzwerten
	 		\item Dosisleistung skaliert in etwa linear mit der Reaktorleistung\\
	 		$\rightarrow$ Leistungs-Reaktoren müssen noch stärker geschirmt werden
	 		\item Abstandsquadratsgesetz wurde bestätigt (so gut das mit so wenigen Messwerten möglich war)
	 		\item da $\rho_{\ddot{U}berschuss} < 1\ \unit{\$}$, besteht weder bei technischen noch bei personellen Fehlern die Möglichkeit, den Reaktor aus dem kritischen in einen prompt überkritischen Zustand zu versetzen
 		\end{itemize}
 	\end{frame}
 
\end{document}
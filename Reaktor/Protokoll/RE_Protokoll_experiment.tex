\section{Durchführung}
    \subsection{Reaktorstart}
    %TODO Was muss vor Inbetriebnahme beachtet werden? Welche Kontrollleuchten gibt es? Wie funktioniert die Steuerung? Wofür gibt es die Schlüssel? Ertönen Warnsignale und wenn ja, wann und warum? 
    Vor der Inbetriebnahme des Reaktors sind mehrere sicherheitsrelevante Prüfungen an dosimetrischen und nicht-dosimetrischen Einrichtungen vorzunehmen. Dies beinhaltet auch die Prüfung von Warnleuchten und Signalgebern auf ihre Funktionalität. Wenn alle Leuchten funktionieren, kann man mit der Überprüfung fortfahren. So muss überprüft werden, ob das Warnsignal bei Erreichen der Grenzwerte bei Leistung (2,2W) und Periodendauer (20s) auch tatsächlich ertönen kann, ob die Reaktorschnellabschaltung (RESA) funktionsfähig ist und alle Signale, betreffend der Stab-, Kernhälften- (KH) und Anfahrneutronenquellenpositionen (ANQ) am Operatortisch ankommen. Man setzt nun den RESA-Schalter am PC zurück, fährt ANQ und untere KH sowie auch die Steuerstäbe in die unterste Endstellung.
    \begin{center}
        \begin{tabular}{l|c|c|c|c}
                \textbf{Ort} & \multicolumn{2}{c|}{\textbf{bei 1\unit{W}}} & \multicolumn{2}{c}{\textbf{bei 2\unit{W}}} \\
       \hline       & $\dot D_\gamma$ [\textit{$\mu Sv/h$}] & $\dot D_n$ [\textit{$\mu Sv/h$}]& $\dot D_\gamma$ [\textit{$\mu Sv/h$}]& $\dot D_n$ [\textit{$\mu Sv/h$}]\\
       \hline   Reaktortankwand ($\approx \unit[0]{m}$) & 12 & 2,5 & 27,5 & 4,3 \\
                Operatortisch ($\approx \unit[3]{m}$) & 1,4 & 0,2 & 2,5 & 0,6 \\
                Ecke ($\approx \unit[6]{m}$) & 0,6 & 0,14 & 0,5 & 0,13 
        \end{tabular}
        \captionof{table}{Dosimetrische Messung von Neutronen und Photonen an verschiedenen Orten und mit unterschiedlichen Leistungen}
        \label{dft:dosi}
    \end{center}
    \subsection{Steuerstabkalibrierung}


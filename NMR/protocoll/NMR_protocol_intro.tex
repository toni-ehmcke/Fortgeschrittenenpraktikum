\section{Introduction}
	\subsection{Motivation}
		\textit{Nuclear Magnetic Resonance} is a physical phenomenon that can be observed while placing an ensemble of nuclei into a static magnetic field and stimulate it with a high-frequent alterning field. A necessary condition for this effect is that the atoms of the sample have a \textit{nuclear spin} different from zero. It is the central concept that is used for \textit{NMR-Spectroscopy}, a standard methodology for the investigation of the structure and interaction of complex molecules and solid state bodies by measuring local magnetic fields, and the \textit{magnetic resonance tomography} which is an imaging technique used in clinical diagnistics for describing the morphilogic and physiologic build-up of tissues and organs. For all of those applications we need to find out some central parameters of particular physical compensation-processes, the so called \textit{relaxation times} $T_1$ and $T_2$. In the following experiment exactly those material-characteristic obserables are determined for an ensemble of $^57$Fe-nuclei. But at first some basic knowledge.
	\subsection{Nuclear Zeeman-Effect}